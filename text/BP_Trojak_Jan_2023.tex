%% This is the ctufit-thesis example file. It is used to produce theses
%% for submission to Czech Technical University, Faculty of Information Technology.
%%
%% Get the newest version from
%% https://gitlab.fit.cvut.cz/theses-templates/FITthesis-LaTeX
%%
%%
%% Copyright 2021, Eliska Sestakova and Ondrej Guth
%%
%% This work may be distributed and/or modified under the
%% conditions of the LaTeX Project Public Licenese, either version 1.3
%% of this license or (at your option) any later version.
%% The latest version of this license is in
%%  https://www.latex-project.org/lppl.txt
%% and version 1.3 or later is part of all distributions of LaTeX
%% version 2005/12/01 or later.
%%
%% This work has the LPPL maintenance status `maintained'.
%%
%% The current maintainer of this work is Ondrej Guth.
%% Contact ondrej.guth@fit.cvut.cz for bug reports.
%% Alternatively, submit bug reports into the tracker at
%% https://gitlab.fit.cvut.cz/theses-templates/FITthesis-LaTeX/issues
%%
%%

%%%%%%%%%%%%%%%%%%%%%%%%%%%%%%%%%%%%%%%%%
% CLASS OPTIONS
% language: czech/english/slovak
% thesis type: bachelor/master/dissertation
%%%%%%%%%%%%%%%%%%%%%%%%%%%%%%%%%%%%%%%%%
\documentclass[czech,bachelor,unicode]{ctufit-thesis}

%%%%%%%%%%%%%%%%%%%%%%%%%%%%%%%%%%
% FILL IN THIS INFORMATION
%%%%%%%%%%%%%%%%%%%%%%%%%%%%%%%%%%
\ctufittitle{Síťová komunikace aplikací v Kubernetes s externími zařízeními v privátní síti} % replace with the title of your thesis
\ctufitauthorfull{Jan Troják}
\ctufitauthorsurnames{Troják}
\ctufitauthorgivennames{Jan}
\ctufitsupervisor{Ing.\ Tomáš\ Vondra,\ Ph.D.}
\ctufitdepartment{Katedra počítačových systémů}
\ctufityear{2023}
\ctufitdeclarationplace{Praze}
\ctufitdeclarationdate{\today} % replace with the date of signature of the declaration
\ctufitabstractCZE{
Tato práce se zabývá možnostmi síťování v systému Kubernetes. Cílem práce bylo rozšířit Kubernetes o možnost adresace a komunikace cloudu se zařízeními v privátních sítích. Známá řešení poskytují pouze komunikaci pomocí vysokoúrovňových protokolů. Cílem bylo nalézt řešení, které by podporovalo protokoly nižších vrstev ISO/OSI.

Práce představuje možnost rozšíření systému Kubernetes o zmíněnou funkcionality síťové komunikace. Tento způsob umožňuje komunikaci pomocí TCP a UDP protokolů se zařízeními v privátních sítích. Představené řešení nabízí flexibilitu použití a nepředstavuje žádná omezení pro standardní použití systému Kubernetes. Řešení je realizováno pomocí zavedených standardů pro rozšiřování systému.  

Výsledky této práce poskytují širší možnosti pro použití systému Kubernetes. Díky tomuto rozšíření je možné lépe využít systém Kubernetes v oblastech testování, smart cities a dalších oblastech pracujících se zařízeními v privátních sítích.

%Výstupem práce je... Výsledek... Co ten operator umi... Zaver (prinosem teto prace)...
}
\ctufitabstractENG{
This thesis explores the networking capabilities of Kubernetes. The aim of the thesis was to extend Kubernetes with the possibility of addressing and communicating with devices in private networks. Known solutions only provide communication using high-level protocols. The goal was to find a solution that would support communication using lower layer ISO/OSI protocols.

This thesis presents the possibility of extending the kubernetes system with the mentioned functionalities of network communication. This method allows communication with devices in private networks using TCP and UDP protocols. The presented solution offers flexibility of use and does not present any limitation restricting standard use of Kubernetes. The solution is implemented using established standards for extending the system.

The results of this work provide wider possibilities for the use of Kubernetes. With this extension, it is possible to make better use of Kubernetes in the areas of testing, smart cities and other areas working with devices in private networks.
}

\ctufitkeywordsCZE{Kubernetes, CNI, network~driver, privátní~síťový~segment, edge~cloud~computing, Kubernetes operátor, síťování, TCP, UDP, K8S}
\ctufitkeywordsENG{Kubernetes, CNI, network~driver, private~network~segment, edge~cloud~computing, Kubernetes operator, networking, TCP, UDP, K8S}
%%%%%%%%%%%%%%%%%%%%%%%%%%%%%%%%%%
% END FILL IN
%%%%%%%%%%%%%%%%%%%%%%%%%%%%%%%%%%

%%%%%%%%%%%%%%%%%%%%%%%%%%%%%%%%%%
% CUSTOMIZATION of this template
% Skip this part or alter it if you know what you are doing.
%%%%%%%%%%%%%%%%%%%%%%%%%%%%%%%%%%

\RequirePackage{iftex}[2020/03/06]
\iftutex % XeLaTeX and LuaLaTeX
    \RequirePackage{ellipsis}[2020/05/22] %ellipsis workaround for XeLaTeX
\else
    \RequirePackage[utf8]{inputenc}[2018/08/11] %this file encoding
    \RequirePackage{lmodern}[2009/10/30] % vector flavor of Computer Modern font
\fi

% hyperlinks
\RequirePackage[pdfpagelayout=TwoPageRight,colorlinks=false,allcolors=decoration,pdfborder={0 0 0.1}]{hyperref}[2020-05-15]

% uncomment the following to hide all hyperlinks
% \RequirePackage[pdfpagelayout=TwoPageRight,hidelinks]{hyperref}[2020-05-15]

\RequirePackage{pdfpages}[2020/01/28]

\setcounter{secnumdepth}{4} % numbering sections; 4: subsubsection



%%%%%%%%%%%%%%%%%%%%%%%%%%%%%%%%%%
% CUSTOMIZATION of this template END
%%%%%%%%%%%%%%%%%%%%%%%%%%%%%%%%%%


%%%%%%%%%%%%%%%%%%%%%%
% DEMO CONTENTS SETTINGS
% You may choose to modify this part.
%%%%%%%%%%%%%%%%%%%%%%
\usepackage{dirtree}
\usepackage{lipsum,tikz}
\usepackage{csquotes}
\usepackage{microtype}
\usepackage[style=iso-numeric]{biblatex}
%\addbibresource{text/bib-database.bib}

\addbibresource{text/bib/all.bib}

\usepackage{listings} % typesetting of sources
\usepackage{float}
\newfloat{lstfloat}{htbp}{lop}[section]
\floatname{lstfloat}{Listing}
% \usepackage{minted} % typesetting of sources

%theorems, definitions, etc.
%\theoremstyle{plain}
%\newtheorem{theorem}{Věta}
%\newtheorem{lemma}[theorem]{Tvrzení}
%\newtheorem{corollary}[theorem]{Důsledek}
%\newtheorem{proposition}[theorem]{Návrh}
%\newtheorem{definition}[theorem]{Definice}
%\theoremstyle{definition}
%\newtheorem{example}[theorem]{Příklad}
%\theoremstyle{remark}
%\newtheorem{note}[theorem]{Poznámka}
%\newtheorem*{note*}{Poznámka}
%\newtheorem{remark}[theorem]{Pozorování}
%\newtheorem*{remark*}{Pozorování}
%\numberwithin{theorem}{chapter}
%theorems, definitions, etc. END
%%%%%%%%%%%%%%%%%%%%%%
% DEMO CONTENTS SETTINGS END
%%%%%%%%%%%%%%%%%%%%%%
\setcounter{biburllcpenalty}{7000}
\setcounter{biburlucpenalty}{8000}
\setlength{\headheight}{13.2pt}% ...at least 51.60004pt
\begin{document} 
\frontmatter\frontmatterinit % do not remove these two commands

\includepdf[pages={1-}]{trojaj12-assignment.pdf} % replace that file with your thesis assignment provided by study office

\thispagestyle{empty}\cleardoublepage\maketitle % do not remove these three commands

\imprintpage % do not remove this command

\tableofcontents % do not remove this command
%%%%%%%%%%%%%%%%%%%%%%
% list of other contents: figures, tables, code listings, algorithms, etc.
% add/remove commands accordingly
%%%%%%%%%%%%%%%%%%%%%%
\listoffigures % list of figures
\begingroup
\let\clearpage\relax
%\listoftables % list of tables
\lstlistoflistings % list of source code listings generated by the listings package
% \listoflistings % list of source code listings generated by the minted package
\endgroup
%%%%%%%%%%%%%%%%%%%%%%
% list of other contents END
%%%%%%%%%%%%%%%%%%%%%%

%%%%%%%%%%%%%%%%%%%
% ACKNOWLEDGMENT
% FILL IN / MODIFY
% This is a place to thank people for helping you. It is common to thank your supervisor.
%%%%%%%%%%%%%%%%%%%
\begin{acknowledgmentpage}
	Chtěl bych poděkovat vedoucímu práce Ing. Tomáši Vondrovi, Ph.D., který velmi pomohl při vzniku této práce ať už věcnými podněty, nebo nasměrováními které vedly k lepším výsledkům. Zároveň bych mu chtěl poděkovat za poskytnutí velké volnosti, díky které jsem se mohl konkrétněji zaměřit na témata mně blízká.
 
    Dále bych chtěl poděkovat Daně Suchomelové, Danielu Hromádkovi a Samuelu Švalichovi za poskytovanou psychickou podporu a prostory pro hacketony, při kterých primárně tato práce vznikla.
 
    V neposlední řadě bych chtěl poděkovat open-source komunitě za vytváření skvělých produktů, které jsou volně dostupné široké veřejnosti. 
\end{acknowledgmentpage} 
%%%%%%%%%%%%%%%%%%%
% ACKNOWLEDGMENT END
%%%%%%%%%%%%%%%%%%%
\lstdefinestyle{mybashstyle}{
  language=bash,
  basicstyle=\small\ttfamily,
  breaklines=true,
  breakatwhitespace=true,
  breakindent=1em,
  %prebreak=\raisebox{0ex}[0ex][0ex]{\ensuremath{\hookleftarrow}},
  postbreak=\mbox{\textcolor{red}{$\hookrightarrow$}\space},
  literate={<~>}{\nobreak\ }1, % Replace <~> with a non-breakable space
}

%%%%%%%%%%%%%%%%%%%
% DECLARATION
% FILL IN / MODIFY
%%%%%%%%%%%%%%%%%%%
% INSTRUCTIONS
% ENG: choose one of approved texts of the declaration. DO NOT CREATE YOUR OWN. Find the approved texts at https://courses.fit.cvut.cz/SFE/download/index.html#_documents (document Declaration for FT in English)
% CZE/SLO: Vyberte jedno z fakultou schvalenych prohlaseni. NEVKLADEJTE VLASTNI TEXT. Schvalena prohlaseni najdete zde: https://courses.fit.cvut.cz/SZZ/dokumenty/index.html#_dokumenty (prohlášení do ZP)
\begin{declarationpage}
Prohlašuji, že jsem předloženou práci vypracoval samostatně a že jsem uvedl veškeré použité
informační zdroje v souladu s Metodickým pokynem o dodržování etických principů při přípravě
vysokoškolských závěrečných prací.

Beru na vědomí, že se na moji práci vztahují práva a povinnosti vyplývající ze zákona č. 121/2000 Sb.,
autorského zákona, ve znění pozdějších předpisů. V souladu s ust. § 2373 odst. 2 zákona č. 89/2012
Sb., občanský zákoník, ve znění pozdějších předpisů, tímto uděluji nevýhradní oprávnění (licenci) k
užití této mojí práce, a to včetně všech počítačových programů, jež jsou její součástí či přílohou a
veškeré jejich dokumentace (dále souhrnně jen „Dílo“), a to všem osobám, které si přejí Dílo užít. Tyto
osoby jsou oprávněny Dílo užít jakýmkoli způsobem, který nesnižuje hodnotu Díla a za jakýmkoli
účelem (včetně užití k výdělečným účelům). Toto oprávnění je časově, teritoriálně i množstevně
neomezené.
\end{declarationpage}
%%%%%%%%%%%%%%%%%%%
% DECLARATION END
%%%%%%%%%%%%%%%%%%%
\newcommand{\term}[1]{{\fontfamily{qcr}\selectfont #1}}
\printabstractpage % do not remove this command
%%%%%%%%%%%%%%%%%%%
% ABBREVIATIONS
% FILL IN / MODIFY
% OR REMOVE ENTIRELY
% List the abbreviations in lexicography order.
%%%%%%%%%%%%%%%%%%%
\chapter{Seznam zkratek}
%grep -rnEoe '[[:upper:]]{2,}' 2>/dev/null | grep -E '^[^:]+tex:' | cut -d ':' -f3 | sort -u
\begin{tabular}{rl}
TCP & Transmission Control Protocol\\
UDP & User Datagram Protocol\\
HTTP & Hypertext Transfer Protocol\\
HIL & Hardware In Loop\\
SIL & Software In Loop\\
OCI & Open Container Initiative\\
CNI & Container Network Interface\\
YAML & YAML Ain't Markup Language\\
JSON & JavaScript Object Notation \\
IP & Internet protocol\\
NAT & Network address translation\\
ISO/OSI & Referenční model ISO/OSI\\
API & rozhraní pro programování aplikací\\
BGP & foo\\
CIDR & foo\\
CNI & foo\\
CRD & foo\\
CRI & foo\\
DNS & foo\\
GRASP & foo\\
ISO & foo\\
LB & foo\\
MQTT & foo\\
OS & foo\\
OSI & foo\\
REST & foo\\
SQL & foo\\
SSL & foo\\
TLS & foo\\
URI & foo\\
VXLAN & foo\\
\end{tabular}
%%%%%%%%%%%%%%%%%%%
% ABBREVIATIONS END
%%%%%%%%%%%%%%%%%%%
\mainmatter\mainmatterinit % do not remove these two commands
%%%%%%%%%%%%%%%%%%%
% THE THESIS
% MODIFY ANYTHING BELOW THIS LINE
%%%%%%%%%%%%%%%%%%%
\textbf{TODO}
\begin{itemize}
    \item Citovat obrazky
    \item 1.3.2 by se vyplatilo vyuzit odrazkoveho seznamu na ty 4 metody reseni, aby to bylo citelnejsi.
    \item Nekde ke konci 1.3.3 chybi obrazek ukazujici Service S a Pod A
\end{itemize}

\chapter*{Úvod}\addcontentsline{toc}{chapter}{Úvod}\markboth{Introduction}{Introduction}
\setcounter{page}{1}

Orchestrační nástroj (zkráceně orchestrátor) je nástroj, který slouží pro ulehčení práce s různými informačními systémy. Často se jedná o technologie složené z malých programů a modulů, které automatizují určité kroky. Ochestrátorů je velká řada z mnoha různých kategorií. Mezi nejznámější orchestrátory kontejnerových aplikací patří kubernetes. Právě tímto orchestrátorem se tato práce zabývá. \cite{goldberg_2019_workflow}

Kubernetes je orchestrační technologie, která poskytuje prostředí pro provoz aplikace na více serverech. Kubernetes tak vytváří a poskytuje jednotné prostředí pro správu aplikací. Toto prostředí se nazývá klastr. Kubernetes podporuje různé způsoby komunikací, mezi aplikacemi a službami v síti klastru, která je tvořena výpočetními uzly. Pro potřeby síťování se v kubernetes používá interní privátní virtuální síť, která je sdílena mezi všemi uzly. Tuto síť mohou využívat všechny objekty, které jsou součástí kubernetes.

Pro propojení vnitřní sítě s okolním světem poskytuje kubernetes standardizovaná řešení. Tyto standardní řešení komunikace, které kubernetes nabízí jsou primárně jednostranné, spoléhají se na komunikaci s veřejnými adresami a nenabízí přímou kontrolu nad tokem dat. Pro oboustrannou komunikaci se zařízeními v privátních sítích, které se nacházejí mimo zmíněnou virtuální síť není technologie v základnu připravena. Toto je značné omezení v případě, že do clusteru je potřeba připojit reálný hardwarový prvek, který nelze přímo integrovat do sítě clusteru. Takovými prvky jsou například jednoduchá zařízení, která sbírají data, různé periferie, testovaná zařízení\ldots Obecně je lze tyto zařízení označit jako externí hardwarové prvky.

Hardwarový prvek který není možné přímo přidat do sítě serverů není jednoduché k kubernetes clusteru připojit. Připojení takového zařízení je velmi náročné a pracné což je je v rozporu s hlavní myšlenkou orchestrace, která má usnadňovat práci, často formou automatizace.

Tato práce se zaměřuje na možnost, jak rozšířit možnosti orchestrátoru kubernetes o možnost adresace a komunikace s hardwarovými periferiemi v privátních sítích při použití kubernetes. Zkoumaná komunikace s hardwarovými prvky bude probíhat pomocí \textit{TCP}, \textit{UDP} a \textit{HTTP} protokolů.

Navržené řešení by mělo být obecné a nezávislé na nestandardním nastavení kubernetes. Zároveň by nemělo nijak ovlivňovat jakékoliv funkcionality kubernetes. 

\newpage

\section{Motivace}
Tato práce vnikla pro potřeby HIL\footnote{HIL (hardware in loop) je technika testování hardwarových zařízení, kde je zařízení testováno v simulovaném prostředí. Simulace prostředí nejčastěji probíhá pomocí matematických modelů, které generují signály pro dané zařízení.} testování v prostředí cloudu. Myšlenkou je nalézt způsob, jakým umožnit testování komunikace různých hardwarových prvků a simulací tak, aby bylo možné tyto prvky a simulace jednoduše kombinovat. Pro tyto účely je zapotřebí umožnit komunikaci mezi kubernetes a zařízeními, které nejsou připojené do internetlvé sítě. Díky integrování HIL testování do prostředí cloudu se zlepší možnosti testování. Zároveň se zjednoduší práce potřebná pro nastavování prostředí, pokud se využije stávajících orchestračních řešení.

\section{Cíle práce}
Cílem této práce je najít způsob, jakým zajistit adresaci a komunikaci s hardwarovými prvky, nacházející se mimo síť kubernetes. Zkoumána bude komunikace protokoly \textit{TCP} , \textit{UDP} a \textit{HTTP}. V případě, že nebude známo žádné řešení, které by splnilo kladené nároky, pak je za cíl návrh a implementace řešení pro výše popsaný problém.

\section{Struktura práce}
Práce je strukturována celkem do tří kapitol. První kapitola představuje základní koncepty systému Kubernetes. Hlavní část bude věnována možnostem síťování, které tento systém nabízí. Informace obsažené v teoretické části slouží jako stavební bloky pro zbytek této práce.

Druhá kapitola je věnována samotnému problému adresace zařízeních v přilehlé interní síti cloudu. Zde je problematika představena převážně na teoretické úrovni. V této části jsou diskutovány možná řešení.

Poslední, třetí kapitola popisuje realizaci řešení představeného v kapitole předchozí. Zde je popsáno prostředí použité při realizaci, konkrétní způsob podpory komunikace a implementace rozšíření systému Kubernetes.

\section{Dohoda se čtenářem}
V této práci se budou často vyskytovat názvy objektů z systému Kubernetes. Tyto objekty budou uvedeny s velkými počátečními písmeny. Toto je zavedená konvence proto, aby se názvy objektů nepletli se slovy běžného jazyka. Tato konvence dává dobrý smysl zejména v anglicky psané literatuře, i přesto, že tato práce je psaná v jazyce českém, bude tato konvence dodržována. Příkladem objektu může být objekt typu Deployment s velkým \'D\'.

Základní pojmy, které jsou v práci použity se nachází na začátku této práce. V průběhu textu může být zaveden nový pojem, který je potřeba podrobněji vysvětlit. V případě zavedení takového pojmu bude pojem zvýrazněn odlišným písmem od okolního textu, aby byl čtenář upozorněn. Ukázkou by mohlo být například slovo \term{kernel}. Kernel (česky jádro) je část operačního systému, která přímo komunikuje s hardwarem počítače.  

V případě, že se jedná o ukázku z příkazové řádky, bude použit specifický blok. Ukázkový blok je uveden ve výpisu kódu \ref{sample:cmd}.

Pokud se příkazy provádějí v různých prostředích, budou prostředí uvedena v hranatých závorkách. Pokud je prostředí jednotné, bude použit symbol podtržítka. Příkazy vždy načínají symbolem \verb|$|, výstupy konzole jsou uvedeny symbolem \verb|>>> |.
\begin{lstfloat}
\begin{lstlisting}[style=mybashstyle,
caption={Ukázkový blok výpisu konzole},
label={sample:cmd}
]
[1]$ ip -4 --brief address show eth0        # this is environment (PC) 1
>>> eth0        UP      192.168.124.176/24
[2]$ ip -4 --brief address show eth0        # this is environment (PC) 2
>>> eth1        DOWN    192.168.124.177/24
\end{lstlisting}
\end{lstfloat}

Příkazy z těchto bloků jsou převeditelné na script pomocí následujícího příkazu \ref{cmd:exec} 
\begin{lstfloat}
\begin{lstlisting}[
style=mybashstyle,
caption={Příkaz převedení výpisu z konzole na spustitelný script},
label={cmd:exec}
]
[_]$ sed 's /^\[[[:digit:]_]\]\$ // ; /^>>>/ d' block.sh | tee script.sh
\end{lstlisting}
\end{lstfloat}

\section{Prostředí a použité verze softwaru}
Veškeré příklady jsou prováděny ve virtuálním stroji na systému \term{CentOS/7}. Virtuální prostředí lze spustit pomocí služby \term{vagarnt}. V případě potřeby lze virtuální prostředí nastavit pomocí následujících příkazů.
\begin{lstfloat}
\begin{lstlisting}[style=mybashstyle,
caption={Nastavení prostředí pomocí Vagrant},
label={sample:vm}
]
[1]$ cat > Vagrantfile <<EOF
Vagrant.configure("2") do |config|
  config.vm.box = "centos/7"
end
EOF
[1]$ vagrant up
[1]$ vagrant ssh
[2]$ hostnamectl
>>>   Static hostname: localhost.localdomain
>>>         Icon name: computer-vm
>>>           Chassis: vm
>>>        Machine ID: d1a6b9d5e7f4af49b5c53c99d86d520b
>>>           Boot ID: 077a2c59fd4545889bb1566fd23d5c58
>>>    Virtualization: kvm
>>>  Operating System: CentOS Linux 7 (Core)
>>>       CPE OS Name: cpe:/o:centos:centos:7
>>>            Kernel: Linux 3.10.0-1127.el7.x86_64
>>>      Architecture: x86-64
\end{lstlisting}
\end{lstfloat}

Pro správné použití je potřeba mít nainstalovaný vagrant a libovolný podporovaný virtualizační nástroj (hypervisor).

Veškeré informace v této práci se vztahují na Kubernetes verzi \verb|1.26.0|.
\textbf{TODO}
\begin{itemize}
    \item Citovat obrazky
    \item 1.3.2 by se vyplatilo vyuzit odrazkoveho seznamu na ty 4 metody reseni, aby to bylo citelnejsi.
    \item Nekde ke konci 1.3.3 chybi obrazek ukazujici Service S a Pod A
\end{itemize}

\chapter*{Úvod}\addcontentsline{toc}{chapter}{Úvod}\markboth{Introduction}{Introduction}
\setcounter{page}{1}

Orchestrační nástroj (zkráceně orchestrátor) je nástroj, který slouží pro ulehčení práce s různými informačními systémy. Často se jedná o technologie složené z malých programů a modulů, které automatizují určité kroky. Ochestrátorů je velká řada z mnoha různých kategorií. Mezi nejznámější orchestrátory kontejnerových aplikací patří kubernetes. Právě tímto orchestrátorem se tato práce zabývá. \cite{goldberg_2019_workflow}

Kubernetes je orchestrační technologie, která poskytuje prostředí pro provoz aplikace na více serverech. Kubernetes tak vytváří a poskytuje jednotné prostředí pro správu aplikací. Toto prostředí se nazývá klastr. Kubernetes podporuje různé způsoby komunikací, mezi aplikacemi a službami v síti klastru, která je tvořena výpočetními uzly. Pro potřeby síťování se v kubernetes používá interní privátní virtuální síť, která je sdílena mezi všemi uzly. Tuto síť mohou využívat všechny objekty, které jsou součástí kubernetes.

Pro propojení vnitřní sítě s okolním světem poskytuje kubernetes standardizovaná řešení. Tyto standardní řešení komunikace, které kubernetes nabízí jsou primárně jednostranné, spoléhají se na komunikaci s veřejnými adresami a nenabízí přímou kontrolu nad tokem dat. Pro oboustrannou komunikaci se zařízeními v privátních sítích, které se nacházejí mimo zmíněnou virtuální síť není technologie v základnu připravena. Toto je značné omezení v případě, že do clusteru je potřeba připojit reálný hardwarový prvek, který nelze přímo integrovat do sítě clusteru. Takovými prvky jsou například jednoduchá zařízení, která sbírají data, různé periferie, testovaná zařízení\ldots Obecně je lze tyto zařízení označit jako externí hardwarové prvky.

Hardwarový prvek který není možné přímo přidat do sítě serverů není jednoduché k kubernetes clusteru připojit. Připojení takového zařízení je velmi náročné a pracné což je je v rozporu s hlavní myšlenkou orchestrace, která má usnadňovat práci, často formou automatizace.

Tato práce se zaměřuje na možnost, jak rozšířit možnosti orchestrátoru kubernetes o možnost adresace a komunikace s hardwarovými periferiemi v privátních sítích při použití kubernetes. Zkoumaná komunikace s hardwarovými prvky bude probíhat pomocí \textit{TCP}, \textit{UDP} a \textit{HTTP} protokolů.

Navržené řešení by mělo být obecné a nezávislé na nestandardním nastavení kubernetes. Zároveň by nemělo nijak ovlivňovat jakékoliv funkcionality kubernetes. 

\newpage

\section{Motivace}
Tato práce vnikla pro potřeby HIL\footnote{HIL (hardware in loop) je technika testování hardwarových zařízení, kde je zařízení testováno v simulovaném prostředí. Simulace prostředí nejčastěji probíhá pomocí matematických modelů, které generují signály pro dané zařízení.} testování v prostředí cloudu. Myšlenkou je nalézt způsob, jakým umožnit testování komunikace různých hardwarových prvků a simulací tak, aby bylo možné tyto prvky a simulace jednoduše kombinovat. Pro tyto účely je zapotřebí umožnit komunikaci mezi kubernetes a zařízeními, které nejsou připojené do internetlvé sítě. Díky integrování HIL testování do prostředí cloudu se zlepší možnosti testování. Zároveň se zjednoduší práce potřebná pro nastavování prostředí, pokud se využije stávajících orchestračních řešení.

\section{Cíle práce}
Cílem této práce je najít způsob, jakým zajistit adresaci a komunikaci s hardwarovými prvky, nacházející se mimo síť kubernetes. Zkoumána bude komunikace protokoly \textit{TCP} , \textit{UDP} a \textit{HTTP}. V případě, že nebude známo žádné řešení, které by splnilo kladené nároky, pak je za cíl návrh a implementace řešení pro výše popsaný problém.

\section{Struktura práce}
Práce je strukturována celkem do tří kapitol. První kapitola představuje základní koncepty systému Kubernetes. Hlavní část bude věnována možnostem síťování, které tento systém nabízí. Informace obsažené v teoretické části slouží jako stavební bloky pro zbytek této práce.

Druhá kapitola je věnována samotnému problému adresace zařízeních v přilehlé interní síti cloudu. Zde je problematika představena převážně na teoretické úrovni. V této části jsou diskutovány možná řešení.

Poslední, třetí kapitola popisuje realizaci řešení představeného v kapitole předchozí. Zde je popsáno prostředí použité při realizaci, konkrétní způsob podpory komunikace a implementace rozšíření systému Kubernetes.

\section{Dohoda se čtenářem}
V této práci se budou často vyskytovat názvy objektů z systému Kubernetes. Tyto objekty budou uvedeny s velkými počátečními písmeny. Toto je zavedená konvence proto, aby se názvy objektů nepletli se slovy běžného jazyka. Tato konvence dává dobrý smysl zejména v anglicky psané literatuře, i přesto, že tato práce je psaná v jazyce českém, bude tato konvence dodržována. Příkladem objektu může být objekt typu Deployment s velkým \'D\'.

Základní pojmy, které jsou v práci použity se nachází na začátku této práce. V průběhu textu může být zaveden nový pojem, který je potřeba podrobněji vysvětlit. V případě zavedení takového pojmu bude pojem zvýrazněn odlišným písmem od okolního textu, aby byl čtenář upozorněn. Ukázkou by mohlo být například slovo \term{kernel}. Kernel (česky jádro) je část operačního systému, která přímo komunikuje s hardwarem počítače.  

V případě, že se jedná o ukázku z příkazové řádky, bude použit specifický blok. Ukázkový blok je uveden ve výpisu kódu \ref{sample:cmd}.

Pokud se příkazy provádějí v různých prostředích, budou prostředí uvedena v hranatých závorkách. Pokud je prostředí jednotné, bude použit symbol podtržítka. Příkazy vždy načínají symbolem \verb|$|, výstupy konzole jsou uvedeny symbolem \verb|>>> |.
\input{text/code/sample_cmd}

Příkazy z těchto bloků jsou převeditelné na script pomocí následujícího příkazu \ref{cmd:exec} 
\input{text/code/cmd_exec}

\section{Prostředí a použité verze softwaru}
Veškeré příklady jsou prováděny ve virtuálním stroji na systému \term{CentOS/7}. Virtuální prostředí lze spustit pomocí služby \term{vagarnt}. V případě potřeby lze virtuální prostředí nastavit pomocí následujících příkazů.
\input{text/code/sample_vm}

Pro správné použití je potřeba mít nainstalovaný vagrant a libovolný podporovaný virtualizační nástroj (hypervisor).

Veškeré informace v této práci se vztahují na Kubernetes verzi \verb|1.26.0|.
\textbf{TODO}
\begin{itemize}
    \item Citovat obrazky
    \item 1.3.2 by se vyplatilo vyuzit odrazkoveho seznamu na ty 4 metody reseni, aby to bylo citelnejsi.
    \item Nekde ke konci 1.3.3 chybi obrazek ukazujici Service S a Pod A
\end{itemize}

\input{text/uvod}
\input{text/teoreticka_cast/text}
\input{text/navrh_reseni/text}
\input{text/implementace/text}

\textbf{TODO}
\begin{itemize}
    \item Citovat obrazky
    \item 1.3.2 by se vyplatilo vyuzit odrazkoveho seznamu na ty 4 metody reseni, aby to bylo citelnejsi.
    \item Nekde ke konci 1.3.3 chybi obrazek ukazujici Service S a Pod A
\end{itemize}

\input{text/uvod}
\input{text/teoreticka_cast/text}
\input{text/navrh_reseni/text}
\input{text/implementace/text}

\textbf{TODO}
\begin{itemize}
    \item Citovat obrazky
    \item 1.3.2 by se vyplatilo vyuzit odrazkoveho seznamu na ty 4 metody reseni, aby to bylo citelnejsi.
    \item Nekde ke konci 1.3.3 chybi obrazek ukazujici Service S a Pod A
\end{itemize}

\input{text/uvod}
\input{text/teoreticka_cast/text}
\input{text/navrh_reseni/text}
\input{text/implementace/text}


\textbf{TODO}
\begin{itemize}
    \item Citovat obrazky
    \item 1.3.2 by se vyplatilo vyuzit odrazkoveho seznamu na ty 4 metody reseni, aby to bylo citelnejsi.
    \item Nekde ke konci 1.3.3 chybi obrazek ukazujici Service S a Pod A
\end{itemize}

\chapter*{Úvod}\addcontentsline{toc}{chapter}{Úvod}\markboth{Introduction}{Introduction}
\setcounter{page}{1}

Orchestrační nástroj (zkráceně orchestrátor) je nástroj, který slouží pro ulehčení práce s různými informačními systémy. Často se jedná o technologie složené z malých programů a modulů, které automatizují určité kroky. Ochestrátorů je velká řada z mnoha různých kategorií. Mezi nejznámější orchestrátory kontejnerových aplikací patří kubernetes. Právě tímto orchestrátorem se tato práce zabývá. \cite{goldberg_2019_workflow}

Kubernetes je orchestrační technologie, která poskytuje prostředí pro provoz aplikace na více serverech. Kubernetes tak vytváří a poskytuje jednotné prostředí pro správu aplikací. Toto prostředí se nazývá klastr. Kubernetes podporuje různé způsoby komunikací, mezi aplikacemi a službami v síti klastru, která je tvořena výpočetními uzly. Pro potřeby síťování se v kubernetes používá interní privátní virtuální síť, která je sdílena mezi všemi uzly. Tuto síť mohou využívat všechny objekty, které jsou součástí kubernetes.

Pro propojení vnitřní sítě s okolním světem poskytuje kubernetes standardizovaná řešení. Tyto standardní řešení komunikace, které kubernetes nabízí jsou primárně jednostranné, spoléhají se na komunikaci s veřejnými adresami a nenabízí přímou kontrolu nad tokem dat. Pro oboustrannou komunikaci se zařízeními v privátních sítích, které se nacházejí mimo zmíněnou virtuální síť není technologie v základnu připravena. Toto je značné omezení v případě, že do clusteru je potřeba připojit reálný hardwarový prvek, který nelze přímo integrovat do sítě clusteru. Takovými prvky jsou například jednoduchá zařízení, která sbírají data, různé periferie, testovaná zařízení\ldots Obecně je lze tyto zařízení označit jako externí hardwarové prvky.

Hardwarový prvek který není možné přímo přidat do sítě serverů není jednoduché k kubernetes clusteru připojit. Připojení takového zařízení je velmi náročné a pracné což je je v rozporu s hlavní myšlenkou orchestrace, která má usnadňovat práci, často formou automatizace.

Tato práce se zaměřuje na možnost, jak rozšířit možnosti orchestrátoru kubernetes o možnost adresace a komunikace s hardwarovými periferiemi v privátních sítích při použití kubernetes. Zkoumaná komunikace s hardwarovými prvky bude probíhat pomocí \textit{TCP}, \textit{UDP} a \textit{HTTP} protokolů.

Navržené řešení by mělo být obecné a nezávislé na nestandardním nastavení kubernetes. Zároveň by nemělo nijak ovlivňovat jakékoliv funkcionality kubernetes. 

\newpage

\section{Motivace}
Tato práce vnikla pro potřeby HIL\footnote{HIL (hardware in loop) je technika testování hardwarových zařízení, kde je zařízení testováno v simulovaném prostředí. Simulace prostředí nejčastěji probíhá pomocí matematických modelů, které generují signály pro dané zařízení.} testování v prostředí cloudu. Myšlenkou je nalézt způsob, jakým umožnit testování komunikace různých hardwarových prvků a simulací tak, aby bylo možné tyto prvky a simulace jednoduše kombinovat. Pro tyto účely je zapotřebí umožnit komunikaci mezi kubernetes a zařízeními, které nejsou připojené do internetlvé sítě. Díky integrování HIL testování do prostředí cloudu se zlepší možnosti testování. Zároveň se zjednoduší práce potřebná pro nastavování prostředí, pokud se využije stávajících orchestračních řešení.

\section{Cíle práce}
Cílem této práce je najít způsob, jakým zajistit adresaci a komunikaci s hardwarovými prvky, nacházející se mimo síť kubernetes. Zkoumána bude komunikace protokoly \textit{TCP} , \textit{UDP} a \textit{HTTP}. V případě, že nebude známo žádné řešení, které by splnilo kladené nároky, pak je za cíl návrh a implementace řešení pro výše popsaný problém.

\section{Struktura práce}
Práce je strukturována celkem do tří kapitol. První kapitola představuje základní koncepty systému Kubernetes. Hlavní část bude věnována možnostem síťování, které tento systém nabízí. Informace obsažené v teoretické části slouží jako stavební bloky pro zbytek této práce.

Druhá kapitola je věnována samotnému problému adresace zařízeních v přilehlé interní síti cloudu. Zde je problematika představena převážně na teoretické úrovni. V této části jsou diskutovány možná řešení.

Poslední, třetí kapitola popisuje realizaci řešení představeného v kapitole předchozí. Zde je popsáno prostředí použité při realizaci, konkrétní způsob podpory komunikace a implementace rozšíření systému Kubernetes.

\section{Dohoda se čtenářem}
V této práci se budou často vyskytovat názvy objektů z systému Kubernetes. Tyto objekty budou uvedeny s velkými počátečními písmeny. Toto je zavedená konvence proto, aby se názvy objektů nepletli se slovy běžného jazyka. Tato konvence dává dobrý smysl zejména v anglicky psané literatuře, i přesto, že tato práce je psaná v jazyce českém, bude tato konvence dodržována. Příkladem objektu může být objekt typu Deployment s velkým \'D\'.

Základní pojmy, které jsou v práci použity se nachází na začátku této práce. V průběhu textu může být zaveden nový pojem, který je potřeba podrobněji vysvětlit. V případě zavedení takového pojmu bude pojem zvýrazněn odlišným písmem od okolního textu, aby byl čtenář upozorněn. Ukázkou by mohlo být například slovo \term{kernel}. Kernel (česky jádro) je část operačního systému, která přímo komunikuje s hardwarem počítače.  

V případě, že se jedná o ukázku z příkazové řádky, bude použit specifický blok. Ukázkový blok je uveden ve výpisu kódu \ref{sample:cmd}.

Pokud se příkazy provádějí v různých prostředích, budou prostředí uvedena v hranatých závorkách. Pokud je prostředí jednotné, bude použit symbol podtržítka. Příkazy vždy načínají symbolem \verb|$|, výstupy konzole jsou uvedeny symbolem \verb|>>> |.
\input{text/code/sample_cmd}

Příkazy z těchto bloků jsou převeditelné na script pomocí následujícího příkazu \ref{cmd:exec} 
\input{text/code/cmd_exec}

\section{Prostředí a použité verze softwaru}
Veškeré příklady jsou prováděny ve virtuálním stroji na systému \term{CentOS/7}. Virtuální prostředí lze spustit pomocí služby \term{vagarnt}. V případě potřeby lze virtuální prostředí nastavit pomocí následujících příkazů.
\input{text/code/sample_vm}

Pro správné použití je potřeba mít nainstalovaný vagrant a libovolný podporovaný virtualizační nástroj (hypervisor).

Veškeré informace v této práci se vztahují na Kubernetes verzi \verb|1.26.0|.
\textbf{TODO}
\begin{itemize}
    \item Citovat obrazky
    \item 1.3.2 by se vyplatilo vyuzit odrazkoveho seznamu na ty 4 metody reseni, aby to bylo citelnejsi.
    \item Nekde ke konci 1.3.3 chybi obrazek ukazujici Service S a Pod A
\end{itemize}

\input{text/uvod}
\input{text/teoreticka_cast/text}
\input{text/navrh_reseni/text}
\input{text/implementace/text}

\textbf{TODO}
\begin{itemize}
    \item Citovat obrazky
    \item 1.3.2 by se vyplatilo vyuzit odrazkoveho seznamu na ty 4 metody reseni, aby to bylo citelnejsi.
    \item Nekde ke konci 1.3.3 chybi obrazek ukazujici Service S a Pod A
\end{itemize}

\input{text/uvod}
\input{text/teoreticka_cast/text}
\input{text/navrh_reseni/text}
\input{text/implementace/text}

\textbf{TODO}
\begin{itemize}
    \item Citovat obrazky
    \item 1.3.2 by se vyplatilo vyuzit odrazkoveho seznamu na ty 4 metody reseni, aby to bylo citelnejsi.
    \item Nekde ke konci 1.3.3 chybi obrazek ukazujici Service S a Pod A
\end{itemize}

\input{text/uvod}
\input{text/teoreticka_cast/text}
\input{text/navrh_reseni/text}
\input{text/implementace/text}


\textbf{TODO}
\begin{itemize}
    \item Citovat obrazky
    \item 1.3.2 by se vyplatilo vyuzit odrazkoveho seznamu na ty 4 metody reseni, aby to bylo citelnejsi.
    \item Nekde ke konci 1.3.3 chybi obrazek ukazujici Service S a Pod A
\end{itemize}

\chapter*{Úvod}\addcontentsline{toc}{chapter}{Úvod}\markboth{Introduction}{Introduction}
\setcounter{page}{1}

Orchestrační nástroj (zkráceně orchestrátor) je nástroj, který slouží pro ulehčení práce s různými informačními systémy. Často se jedná o technologie složené z malých programů a modulů, které automatizují určité kroky. Ochestrátorů je velká řada z mnoha různých kategorií. Mezi nejznámější orchestrátory kontejnerových aplikací patří kubernetes. Právě tímto orchestrátorem se tato práce zabývá. \cite{goldberg_2019_workflow}

Kubernetes je orchestrační technologie, která poskytuje prostředí pro provoz aplikace na více serverech. Kubernetes tak vytváří a poskytuje jednotné prostředí pro správu aplikací. Toto prostředí se nazývá klastr. Kubernetes podporuje různé způsoby komunikací, mezi aplikacemi a službami v síti klastru, která je tvořena výpočetními uzly. Pro potřeby síťování se v kubernetes používá interní privátní virtuální síť, která je sdílena mezi všemi uzly. Tuto síť mohou využívat všechny objekty, které jsou součástí kubernetes.

Pro propojení vnitřní sítě s okolním světem poskytuje kubernetes standardizovaná řešení. Tyto standardní řešení komunikace, které kubernetes nabízí jsou primárně jednostranné, spoléhají se na komunikaci s veřejnými adresami a nenabízí přímou kontrolu nad tokem dat. Pro oboustrannou komunikaci se zařízeními v privátních sítích, které se nacházejí mimo zmíněnou virtuální síť není technologie v základnu připravena. Toto je značné omezení v případě, že do clusteru je potřeba připojit reálný hardwarový prvek, který nelze přímo integrovat do sítě clusteru. Takovými prvky jsou například jednoduchá zařízení, která sbírají data, různé periferie, testovaná zařízení\ldots Obecně je lze tyto zařízení označit jako externí hardwarové prvky.

Hardwarový prvek který není možné přímo přidat do sítě serverů není jednoduché k kubernetes clusteru připojit. Připojení takového zařízení je velmi náročné a pracné což je je v rozporu s hlavní myšlenkou orchestrace, která má usnadňovat práci, často formou automatizace.

Tato práce se zaměřuje na možnost, jak rozšířit možnosti orchestrátoru kubernetes o možnost adresace a komunikace s hardwarovými periferiemi v privátních sítích při použití kubernetes. Zkoumaná komunikace s hardwarovými prvky bude probíhat pomocí \textit{TCP}, \textit{UDP} a \textit{HTTP} protokolů.

Navržené řešení by mělo být obecné a nezávislé na nestandardním nastavení kubernetes. Zároveň by nemělo nijak ovlivňovat jakékoliv funkcionality kubernetes. 

\newpage

\section{Motivace}
Tato práce vnikla pro potřeby HIL\footnote{HIL (hardware in loop) je technika testování hardwarových zařízení, kde je zařízení testováno v simulovaném prostředí. Simulace prostředí nejčastěji probíhá pomocí matematických modelů, které generují signály pro dané zařízení.} testování v prostředí cloudu. Myšlenkou je nalézt způsob, jakým umožnit testování komunikace různých hardwarových prvků a simulací tak, aby bylo možné tyto prvky a simulace jednoduše kombinovat. Pro tyto účely je zapotřebí umožnit komunikaci mezi kubernetes a zařízeními, které nejsou připojené do internetlvé sítě. Díky integrování HIL testování do prostředí cloudu se zlepší možnosti testování. Zároveň se zjednoduší práce potřebná pro nastavování prostředí, pokud se využije stávajících orchestračních řešení.

\section{Cíle práce}
Cílem této práce je najít způsob, jakým zajistit adresaci a komunikaci s hardwarovými prvky, nacházející se mimo síť kubernetes. Zkoumána bude komunikace protokoly \textit{TCP} , \textit{UDP} a \textit{HTTP}. V případě, že nebude známo žádné řešení, které by splnilo kladené nároky, pak je za cíl návrh a implementace řešení pro výše popsaný problém.

\section{Struktura práce}
Práce je strukturována celkem do tří kapitol. První kapitola představuje základní koncepty systému Kubernetes. Hlavní část bude věnována možnostem síťování, které tento systém nabízí. Informace obsažené v teoretické části slouží jako stavební bloky pro zbytek této práce.

Druhá kapitola je věnována samotnému problému adresace zařízeních v přilehlé interní síti cloudu. Zde je problematika představena převážně na teoretické úrovni. V této části jsou diskutovány možná řešení.

Poslední, třetí kapitola popisuje realizaci řešení představeného v kapitole předchozí. Zde je popsáno prostředí použité při realizaci, konkrétní způsob podpory komunikace a implementace rozšíření systému Kubernetes.

\section{Dohoda se čtenářem}
V této práci se budou často vyskytovat názvy objektů z systému Kubernetes. Tyto objekty budou uvedeny s velkými počátečními písmeny. Toto je zavedená konvence proto, aby se názvy objektů nepletli se slovy běžného jazyka. Tato konvence dává dobrý smysl zejména v anglicky psané literatuře, i přesto, že tato práce je psaná v jazyce českém, bude tato konvence dodržována. Příkladem objektu může být objekt typu Deployment s velkým \'D\'.

Základní pojmy, které jsou v práci použity se nachází na začátku této práce. V průběhu textu může být zaveden nový pojem, který je potřeba podrobněji vysvětlit. V případě zavedení takového pojmu bude pojem zvýrazněn odlišným písmem od okolního textu, aby byl čtenář upozorněn. Ukázkou by mohlo být například slovo \term{kernel}. Kernel (česky jádro) je část operačního systému, která přímo komunikuje s hardwarem počítače.  

V případě, že se jedná o ukázku z příkazové řádky, bude použit specifický blok. Ukázkový blok je uveden ve výpisu kódu \ref{sample:cmd}.

Pokud se příkazy provádějí v různých prostředích, budou prostředí uvedena v hranatých závorkách. Pokud je prostředí jednotné, bude použit symbol podtržítka. Příkazy vždy načínají symbolem \verb|$|, výstupy konzole jsou uvedeny symbolem \verb|>>> |.
\input{text/code/sample_cmd}

Příkazy z těchto bloků jsou převeditelné na script pomocí následujícího příkazu \ref{cmd:exec} 
\input{text/code/cmd_exec}

\section{Prostředí a použité verze softwaru}
Veškeré příklady jsou prováděny ve virtuálním stroji na systému \term{CentOS/7}. Virtuální prostředí lze spustit pomocí služby \term{vagarnt}. V případě potřeby lze virtuální prostředí nastavit pomocí následujících příkazů.
\input{text/code/sample_vm}

Pro správné použití je potřeba mít nainstalovaný vagrant a libovolný podporovaný virtualizační nástroj (hypervisor).

Veškeré informace v této práci se vztahují na Kubernetes verzi \verb|1.26.0|.
\textbf{TODO}
\begin{itemize}
    \item Citovat obrazky
    \item 1.3.2 by se vyplatilo vyuzit odrazkoveho seznamu na ty 4 metody reseni, aby to bylo citelnejsi.
    \item Nekde ke konci 1.3.3 chybi obrazek ukazujici Service S a Pod A
\end{itemize}

\input{text/uvod}
\input{text/teoreticka_cast/text}
\input{text/navrh_reseni/text}
\input{text/implementace/text}

\textbf{TODO}
\begin{itemize}
    \item Citovat obrazky
    \item 1.3.2 by se vyplatilo vyuzit odrazkoveho seznamu na ty 4 metody reseni, aby to bylo citelnejsi.
    \item Nekde ke konci 1.3.3 chybi obrazek ukazujici Service S a Pod A
\end{itemize}

\input{text/uvod}
\input{text/teoreticka_cast/text}
\input{text/navrh_reseni/text}
\input{text/implementace/text}

\textbf{TODO}
\begin{itemize}
    \item Citovat obrazky
    \item 1.3.2 by se vyplatilo vyuzit odrazkoveho seznamu na ty 4 metody reseni, aby to bylo citelnejsi.
    \item Nekde ke konci 1.3.3 chybi obrazek ukazujici Service S a Pod A
\end{itemize}

\input{text/uvod}
\input{text/teoreticka_cast/text}
\input{text/navrh_reseni/text}
\input{text/implementace/text}



\appendix\appendixinit % do not remove these two commands


 % include `appendix.tex' from `text/' subdirectory
\backmatter % do not remove this command
\printbibliography % print out the BibLaTeX-generated bibliography list
\chapter{Obsah přiloženého archivu}
	\dirtree{%
.1 .
.1 ansible/\DTcomment{definice prostředí pomocí ansible}.
.2 playbook/\DTcomment{definice prostředí pomocí ansible}.
.2 inventory/\DTcomment{seznam hosts pro andisble}.
.2 vars/\DTcomment{proměnné pro ansible playbooks}.
.1 code/\DTcomment{zdrojové kódy}.
.2 EdgeOperator/\DTcomment{zdrojové kódy operatoru}.
.3 EdgeOperator/\DTcomment{operátor}.
.3 EdgeOperator.Tests/\DTcomment{unit testy}.
.3 EdgeOperator.sln.
.2 sampleSrvers/\DTcomment{zdrojové kódy pomocných programů pro testování}.
.3 tcpServer/.
.3 udpServer/.
.1 doc/.\DTcomment{návod na instalaci prostředí a instalaci operatoru}.
.1 manifests/\DTcomment{manifesty pro Kubenretes}.
.2 lab/.
.2 operator/.
.1 scripts/.\DTcomment{pomocná scripty pro instalaci prostředí}.
.1 text/\DTcomment{text bakalářské práce}.
.1 vagrant/\DTcomment{adresář obsahující definice virtuálního prostředí}.
.1 .github\DTcomment{definice pipelines na GitHub}.
.1 readme.md\DTcomment{stručný popis repositáře}.
} % include `medium.tex' from `text/' subdirectory
\end{document}
