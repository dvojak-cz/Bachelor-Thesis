\chapter*{Úvod}\addcontentsline{toc}{chapter}{Úvod}\markboth{Úvod}{Úvod}
\setcounter{page}{1}

Orchestrační nástroj (zkráceně orchestrátor) je nástroj, který slouží pro ulehčení práce s různými informačními systémy. Často se jedná o technologii složenou z malých programů a modulů, které automatizují a ulehčují práci. Ochestrátorů je velká řada z mnoha různých kategorií. Mezi nejznámější orchestrátory kontejnerových aplikací patří Kubernetes. Právě tímto orchestrátorem se tato práce zabývá. \cite{goldberg_2019_workflow}

Kubernetes je orchestrační technologie, která poskytuje prostředí pro provoz aplikací na více serverech. Kubernetes tak vytváří a poskytuje jednotné prostředí pro správu aplikací, nazývané cloud. Kubernetes podporuje různé způsoby komunikací mezi aplikacemi a službami v síti cloudu. Toto prostředí je tvořeno více výpočetními uzly. Pro potřeby síťování Kubernetes používá interní privátní virtuální síť, která je sdílena mezi všemi uzly systému. Tuto síť mohou využívat všechny objekty, které jsou součástí daného systému Kubernetes.

Pro propojení vnitřní sítě s okolním světem poskytuje Kubernetes standardní řešení. Tato standardní řešení komunikace, které systém nabízí jsou primárně jednostranné, spoléhají se na komunikaci s veřejnými adresami a nenabízí přímou kontrolu nad tokem dat. Pro spolehlivou oboustrannou komunikaci se zařízeními v privátních sítích, které se nacházejí mimo zmíněnou virtuální síť, není technologie v základu připravena. To je značné omezení v případě, že do systému je potřeba připojit reálný hardwarový prvek, který nelze přímo integrovat do sítě Kubernetes. Takovými prvky jsou například jednoduchá zařízení, která sbírají data, různé periferie, testovaná zařízení apod. Obecně je lze tato zařízení označit jako externí hardwarové prvky.

Tato práce se zaměřuje na to, jak rozšířit možnosti orchestrátoru Kubernetes o možnost adresace a komunikace s hardwarovými zařízeními v privátních sítích. Zkoumaná komunikace s hardwarovými prvky bude probíhat pomocí \textit{TCP}, \textit{UDP} a \textit{HTTP} protokolů.

Navržené řešení by mělo být obecné a nezávislé na nestandardním nastavení Kubernetes. Zároveň by nemělo nijak ovlivňovat jakékoliv funkcionality systému. 

\newpage

\section{Motivace}
Tato práce vnikla pro potřeby HIL\footnote{HIL (hardware in loop) je technika testování hardwarových zařízení, kde je zařízení testováno v simulovaném prostředí. Simulace prostředí nejčastěji probíhá pomocí matematických modelů, které generují signály pro daná zařízení.} testování v prostředí cloudu. Hlavní myšlenkou je nalézt způsob, jak umožnit testování komunikace různých hardwarových prvků a simulací tak, aby bylo možné prvky a simulace jednoduše kombinovat. Pro tyto účely je zapotřebí umožnit komunikaci mezi Kubernetes a zařízeními, které nejsou připojené do interní sítě. Díky integraci HIL testování do prostředí cloudu se zlepší možnosti testování. Zároveň se zjednoduší práce potřebná pro nastavování prostředí.

\section{Cíle práce}
Cílem této práce je nalézt způsob, jak zajistit adresaci a komunikaci s hardwarovými prvky, nacházejícími se mimo interní síť Kubernetes. Zkoumána bude komunikace pomocí protokolů \textit{TCP} , \textit{UDP} a \textit{HTTP}. V případě, že nebude známo žádné řešení, které by splnilo kladené nároky, pak je za cíl považován návrh a implementace řešení pro výše popsaný problém.

\section{Struktura práce}
Práce je strukturována do tří kapitol. První kapitola představuje základní koncepty systému Kubernetes. Hlavní část bude věnována možnostem síťování, které tento systém nabízí. Informace obsažené v teoretické části slouží jako stavební bloky pro zbytek této práce.

Druhá kapitola je věnována samotnému problému adresace zařízeních v přilehlé privátní síti. Zde je problematika představena převážně na teoretické úrovni. V této části jsou diskutovány možná řešení umožňující zkoumanou komunikaci.

Poslední, třetí kapitola popisuje realizaci řešení představeného v kapitole předchozí. Zde je popsáno prostředí použité při implementaci, konkrétní způsob podpory komunikace a implementace rozšíření systému Kubernetes.

\section{Dohoda se čtenářem}
V této práci se budou často vyskytovat názvy objektů ze systému Kubernetes. Tyto názvy budou uvedeny s velkými počátečními písmeny. Toto je zavedená konvence proto, aby se názvy objektů nepletly se slovy běžného jazyka. Tato konvence dává dobrý smysl zejména v anglicky psané literatuře. I přesto, že tato práce je psaná v jazyce českém, bude tato konvence dodržována. Příkladem objektu může být objekt typu Deployment s velkým počátečním \uv{D}.

V případě, že je v práci uveden výpis z příkazové řádky, nebo část zdrojového kódu, bude použit specifický blok. Ukázkový blok je uveden ve výpisu kódu \ref{sample:cmd}.

Pokud se příkazy v ukázkách provádějí v různých prostředích, budou prostředí uvedena v hranatých závorkách. Pokud je prostředí jednotné, bude použit symbol podtržítka. Příkazy vždy začínají symbolem \verb|$|, výstupy konzole jsou uvedeny symbolem \verb|>>> |.
\begin{lstfloat}
\begin{lstlisting}[style=mybashstyle,
caption={Ukázkový blok výpisu konzole},
label={sample:cmd}
]
[1]$ ip -4 --brief address show eth0        # this is environment (PC) 1
>>> eth0        UP      192.168.124.176/24
[2]$ ip -4 --brief address show eth0        # this is environment (PC) 2
>>> eth1        DOWN    192.168.124.177/24
\end{lstlisting}
\end{lstfloat}

Příkazy z těchto bloků jsou převeditelné na spustitelný script pomocí příkazu. který je uveden v ukázce \ref{cmd:exec}.
\begin{lstfloat}
\begin{lstlisting}[
style=mybashstyle,
caption={Příkaz převedení výpisu z konzole na spustitelný script},
label={cmd:exec}
]
[_]$ sed 's /^\[[[:digit:]_]\]\$ // ; /^>>>/ d' block.sh | tee script.sh
\end{lstlisting}
\end{lstfloat}

\section{Prostředí a použité verze softwaru}
Veškeré příklady jsou prováděny ve virtuálním stroji na systému \term{CentOS/7}. Virtuální prostředí lze spustit pomocí služby \term{Vagarnt}. V případě potřeby lze virtuální prostředí nastavit pomocí následujícího příkazu, který je uveden ve výpisu \ref{sample:vm}.
\begin{lstfloat}
\begin{lstlisting}[style=mybashstyle,
caption={Nastavení prostředí pomocí Vagrant},
label={sample:vm}
]
[1]$ cat > Vagrantfile <<EOF
Vagrant.configure("2") do |config|
  config.vm.box = "centos/7"
end
EOF
[1]$ vagrant up
[1]$ vagrant ssh
[2]$ hostnamectl
>>>   Static hostname: localhost.localdomain
>>>         Icon name: computer-vm
>>>           Chassis: vm
>>>        Machine ID: d1a6b9d5e7f4af49b5c53c99d86d520b
>>>           Boot ID: 077a2c59fd4545889bb1566fd23d5c58
>>>    Virtualization: kvm
>>>  Operating System: CentOS Linux 7 (Core)
>>>       CPE OS Name: cpe:/o:centos:centos:7
>>>            Kernel: Linux 3.10.0-1127.el7.x86_64
>>>      Architecture: x86-64
\end{lstlisting}
\end{lstfloat}

Pro správné použití je potřeba mít nainstalovaný Vagrant a libovolný podporovaný virtualizační nástroj (hypervisor).

Veškeré informace v této práci se vztahují k verzi Kubernetes \verb|1.26.0|.