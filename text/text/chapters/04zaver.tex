\chapter{Závěr}
V závěru této bakalářské práce se podíváme na hodnocení dosažených výsledků a přínosů představeného řešení. Hlavním cílem této práce bylo rozšířit funkcionalitu Kubernetes o komunikaci se zařízeními mimo interní síť cloudu. Konkrétně se jednalo o zařízení v privátních síťových segmentech, která jsou propojená s klastrem pomocí jednoho či více pracovních uzlů. Tato práce byla motivována potřebami při testování HIL a integrací testování do oblasti cloudu. Nalezení řešení a následné rozšíření funkcionality Kubernetes umožňuje z prostředí cloudu testovat zařízení, které není možné připojit přímo do Kubernetes.    

První část práce byla zaměřena na porozumění stávajících řešení síťování a standardů použitých pro síťování. Zároveň částečně ukázala možnost implementací síťování v systému Kubernetes. Tato část pomohla převážně k pochopeni problematiky a ujištění se, že systém Kubernetes opravdu nenabízí dostatečnou možnost adresace do přilehlých privátních síťových segmentů.

V další části se práce zaměřila na řešení daného problému. V této části byly deklarovány požadavky pro hledané řešení. Díky těmto požadavkům bylo možné hodnotit jednotlivá možná řešení, identifikovat jejich nedostatky a přínosy. Jako první byla diskutována možnost využití stávajícího projektu \textit{kube edge}. Tento projekt nabízí možnost řešení problému, která je omezena na MQTT případně HTTP protokol. I přesto, že tato technologie nesplňuje všechny nastavené požadavky pro řešení, jedná se o velmi ambiciozní projekt. Je určitá pravděpodobnost, že projekt bude zkoumanou funkcionalitu implementovat a proto je velmi vhodné pozorovat vývoj tohoto projektu. Druhá zkoumaná možnost se zabývala možností využití proxy. To bylo určeno jako vhodné řešeni problému. Navržené řešení pomocí technologie proxy bylo do značné míry umožněné díky aktivitám komunity \textit{Network Plumbing Working Group}.

Poslední část této práce byla zaměřena na rozšířen systému Kubernetes o možnost diskutovaného způsobu síťování. Tato kapitola popisuje konkretní řešení a návrh rozšíření. Zároveň zmiňuje prostředí, které bylo pro vývoj a testováni použito, společně s důvody volby řešení.

Výsledek práce poskytuje způsob, pro adresaci zařízeních v přilehlých privátní sítích z interní sítě Kubernetes. Konkretní řešení je navrženo tak, aby umožnilo adresaci převážně směrem z cloudu do privátní sítě. Tento směr dává pro reálně použití největší smysl. Řešení je navrženo tak, aby bylo jednoduše rozšiřitelné a podporovalo zavedené standardy v systému Kubernetes.

\bigskip

Výsledky této práce přinesou nové možnosti pro použití orchestrátoru v oblasti testování. Zároveň mohou pomoci při práci s IoT zařízeními v oblastech smart cities a podobných oblastech, které se dotýkají tématu edge cloud computing.

Tato práce zkoumala přímo komunikaci a adresaci s externími zařízeními. Možná rozšíření, jako podpora šifrování nebo autentizace nejsou součástí této práce. Určitě se jedná zajímavé oblasti, které dává smysl dále zkoumat. Mezi další možná rozšíření patří integrace s dalšími nástroji (dynamická konfigurace firewallu atd.). Pro potřeby HIL testování dává smysl se zabývat spolehlivostí a efektivitou daného řešení.

Celkově lze konstatovat, že tato bakalářská práce dosáhla svého hlavního úkolu -- posílit schopnosti Kubernetes v oblasti komunikace s externími zařízeními, která se nacházejí mimo interní síť cloudu. Díky výsledkům této práce se otevírají nové možnosti pro efektivní řešení problémů a významně se rozšiřuje spektrum aplikací systému Kubernetes.
