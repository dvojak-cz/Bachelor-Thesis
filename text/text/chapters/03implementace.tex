\chapter{Návrh řešení a implementace}
\begin{chapterabstract}
Tato kapitola popisuje návrh a implementaci řešení problému z předchozí kapitoly. Cílem této kapitoly navrhnout a vytvořit modul pro systému Kubernetes, který umožní jednoduchých způsobem propojit interní síť Kubernets a přilehlou privátní sítě. Součástí této kapitoly je ukázka a nastínění způsobu použití modulu.
\end{chapterabstract}

\section{Prostředí}\label{prostredi}
Celý vývoj, testování a ukázka probíhá ve prostředí virtuální. Díky tomuto je možné řešení vyzkoušet bez nutnosti hardwaru. Při definici prostředí byl kladen důraz na jeho snadnou replikovatelnost. Veškeré definice a konfigurace jsou dostupné v přiloženém archivu, který je kopií \href{veřejného repozitáře}{veřejného repositáře} na platformě GitHub. Přiložené médium zároveň poskytuje dokumentaci a návod k rozběhnuti samotného prostředí. Tato dokumentace je uložena v adresáři \verb|doc|.\footnote{Tato dokumentace prostředí je dosupná i na adrese \href{https://bt.project.dvojak.cz/}{bt.project.dvojak.cz}}

Vzhledem k tomu, že předmětem práce není instalace případně konfigurace prostředí a návod na zprovoznění je obsažen v přiložených souborech, bude zde postup nastavení prostředí zmíněn jen stručně.

Následující nastavení probíhala na sytému s operačním systémem \textit{debian-11}.
\subsection{Virtuální stroje}
 Pro virtualizaci byl využit hypervizor \textit{virtualbox-6.1} od společnosti Oracle. Virtualbox byl ovládán a konfigurován pomocí nástroje \textit{vagrant-2.3.4} od společnosti HashiCorp. Použité technologie a nástroj byly zvoleny, pro jednoduchou ovladatelnost a možnost definice prostředí formou zdrojového kódu. Díky tomuto je možné deklarativně definovat prostředí, které je snadno nasatvitelné a replikovatelné. 

Pro účely vývoje byly vytvořeny celkem čtyři virtuální stroje. Každé z těchto zařízení vychází z definice vagrant boxu \verb|ubuntu/focal64|. Stroje \textit{kmaster}, \textit{kworker1}, \textit{kedge1} jsou servery, na kterých bude provozován kubenretes. Všechny tyto stroje jsou součástí jednoho klastru. Kube-edge je jediný virtuální stroj, který je připojen do dvou síťových segmentů. Tento server plní roli edge~uzlu\footnote{Edge~uzel označuje stroj, který propojují privátní síť klastru a vnitřní síť Kubernetes.}. Čtvrtým virtuální strojem je device01-01. Tento stroj reprezentuje zařízení umístěné v privátní síti. Zařízení device01-01 není součástí sítě Kuberetes. Konfigurace je znázorněna na schématu \ref{fig:lab} níže.

Kompletní konfigurace virtuálního prostředí je definována v souboru \verb|vagrant/Vagrantfile|. Tento soubor je parametrizovatelný pro potřeby úpravy prostředí. 
\begin{figure}[ht]
    \centering
    \includegraphics[width=0.7\textwidth]{images/lab.pdf}
    \caption{Caption}
    \label{fig:lab}
\end{figure}

\subsection{Konfigurace serverů}
Aby bylo možné provozovat kubenretes na serverch, je nutné servery nakonfigurovat a nainstalovat potřebné programy pro provoz klastru. Konfigurace a instalace je definovaná pomocí ansible playbooku\footnote{Playbook označuje soubor definující úlohy pro konfiguraci pomocí ansible.}. Playbook pro konfiguraci serverů je uložen v souboru \verb|ansible/playbook/infra.yaml|.

Konfigurace a instalace se skládala z následujících kroků, jdoucích po sobě.
\begin{enumerate}
    \item Nastavení firewall
    \item Vypnutí SWAP (odkládací prostor na disku)
    \item Zavedení modulů kernelu pro potřeby Kubernetes a Containerd
    \item Nastavení parametrů kernelu pro Containerd
    \item Instalace Containerd
    \item Konfigurace Containerd a nastavení podpory Cgrups
    \item Instalace Kubernetes
    \item Vytvoření uživatele \textit{kube} pro ovládání Kubernetes
    \item Nastavení parametrů kubelet\label{en:ip_set}
    \item Nastavení parametru kernelu pro přeposílání ARP dotazu\label{en:arp_forward}
\end{enumerate}
Jedná se o standardní instalaci Kubernetes s Containerd poskytující běhové prostředí kontejnerů dle OCI. Předposlední bod z výše uvedených je přidán, aby bylo možné provozovat systém Kubernetes na serverech s více sítovými rozhraními. Poslední krok \ref{en:arp_forward} jesoučástí konfigurace pro případné testování podpory obousměrnosti komunikace. Tato možnost je krátce zmíněna na konci sekce \ref{ukazkaProxy}.

\subsection{Konfigurace zařízeních}
Pro možnost testování je nutné nastavit i virtuálním stroj device01-01 tak, aby provozoval aplikace komunikující za pomocí TCP, UDP a HTTP protokolu. Pro jednoduchost provozování aplikací je použit docker. Konfigurace a instalace Dockeru. Konfigurace a instalace byla provedena pomocí ansible playbooku \verb|ansible/playbook/device.yaml|.
\subsection{Instalace a konfigurace Kubernetes}
Instalace Kubernetes je prováděna pomocí nástroje \textit{kubeadm}. Kubeadm umožňuje jednoduchou instalaci klastru Kubernetes. Pro konfiguraci instalace klastru lze využít parametry, které kubeadm nabízí. Důležitými parametru pro instalaci klastru jsou \verb|--apiserver-advertise-address| nastavující primární IP adresu API serveru, \verb|--pod-network-cidr| nastavující proměnnou \textit{podCIDR} a \verb|--apiserver-cert-extra-sans| definující parametry pro generování TLS certifikátu.
 
Součástí instalace Kubernetes je instalace samotného CNI, který bude použit pro síťování uvnitř klastru. Pro ukázkové prostředí byl použit modul \textit{flannel}. Jedná se o jednoduchý a často používaný CNI Kubernetes plugin.\footnote{Řešení podporuje i použití jiných CNI pluginy. Funkcionalita byla testována pouze na modulech Flannel a Calico}.

\subsection{Výsledné prostředí}
Výsledné prostředí poskytuje dobré možnosti pro testování a realizaci samotného návrh rozšíření. Následující části textu budou vyvíjeny a prezentovány na prostředí odpovídající instalaci popsané výše. Postup rozběhnutí prostředí přesně odpovídá postupu, který je popsán v přiloženém archivu. 

\section{Návrh řešení a implemetace}
Před implementací samotného řešení je nutné provést samotný návrh tak, aby splňoval již deklarované požadavky. Při návrhu budeme vycházet z probíraného řešení pomocí proxy, které je popsáno v předchozí v sekci \ref{sec:req} v minulé kapitole. Návrhu bude používat již zmíněný a probraný koncept Podu s více síťovými rozhraními. Konfiguraci Podu s více rozhraními bude docíleno za použití implementace standardu \textit{Kubernetes Network Custom Resource Definition De-facto Standard}.

Návrh řešení se bude skládat z dvou hlavních částí. V první části bude diskutován výběr aplikace pro provozování služby proxy. Druhá část bude popisovat konkrétní návrh konfigurace síťování v podu. Na závěr bude provedena ukázka navrženého řešení.

\subsection{Vymezení implementace}
V předešlé kapitole jsou deklarovány požadavky na hledané řešení problému a následné rozšíření funkcionality Kubernetes. Návrh si klade nároky každý z těchto požadavků splnit. Zároveň se snaží o to, aby nepřinášelo žádná výrazná omezení pro případné použití. Veškeré koncepty a myšlenky, které budou dále popsány budou klást důraz na snadnou konfigurovatelnost a případnou rozšiřitelnost.

I přesto tuto snahu dává smysl se omezit na určitý způsob použití výsledného rozšíření. Tato omezení jsou použita zejména pro umožnění automatizace práce formou operátoru. Automatizace bude popsána v sekci \ref{sec:operator} zabývající se rozšířením Kubernets. Pro každé zmíněné omezení bude nabídnuta alternativa, která dané omezení umožňuje realizovat. 

\subsubsection*{Podpora protokolů}
V požadavcích pro řešení problému je zmíněná podpora TCP, UDP a HTTP protokolu. TCP a UDP protokoly se nachází na stejné abstrakční vrstvě dle ISO/OSI modelu. Protokol HTTP se nachází na vrstvě sedmé, nejvyšší abstrakční vrstvě. Tento protokol se spoléhá na zmíněné protokoly TCP a UDP. Dnes používané verze HTTP (verze v1 a v2) jsou provozovány za pomocí TCP protokolu. Verze v3 protokolu HTTP je stále vyvíjena, dnešní návrh počítá, že protokol bude využívat QUIC protokol, který je postaven na UDP. \cite{fesl_2021_aplikan}.

Pro potřeby práce bude navrhované řešení zaměřeno převážně na komunikaci pomocí protokolů TCP a UDP. Toto je možné, jelikož nalezení řešní pro tyto dva protokoly implikuje i podporu HTTP. Přímo podpora protokolu HTTP bude diskutována v sekci níže. Jeho podpora může přinést zajímavé možnosti rozšíření funkcionality. Jeho podpora ale výrazně komplikuje konfiguraci a návrh řešení a pro automatizaci práce není vhodná.

\subsubsection*{Jednosměrnost komunikace}\label{vymezeni:1smer}
Již v předchozí kapitola byla zaměřena převážně na jednostrannou komunikaci směrem z interní sítě clastru do sítě privátní. Jednosměrnou komunikaci se myslí pouze směr, kterým je vyslán první packet komunikace. Takto zůstane podporována oboustrannost komunikace pomocí TCP v případě, že komunikace je zahájena z vnitřní sítě klastru. Směr adresace z privátní sítě do sítě klastru je velmi specifický případ užití a proto nebude přímo součástí implementace. Možnost podpory toho směru bude diskutována níže.

\subsection{Výběr proxy}
Aplikaci poskytující službu proxy bude použita pro samotné propojení sítí, proto je její výběr velmi důležitý. Tato služba by měla slňovat následující požadavky: možnost provozování v kontejneru, snadná konfigurovatelnost, možnost provozování více proxy najednou.

Jelikož je služba proxy dlouho známí a často používaný koncept, aplikací poskytující tuto službu je celá řada. Z tohoto důvodu nedává smysl vyvíjet nové vlastní řešení. Namísto toho lze využít již existujících implementací. Mezi známe implementace, které splňují výše deklarované požadavky patří například: HAProxy, Nginx, Envoy, SoCat. Právě SoCat byl vybrán jako vhodná pro provoz služby proxy.

SoCat je nástroj vyvinutý společností redhead pro přenost dat. Tento nástroj je navržen tak, aby byl velmi jednoduchý a zároveň vysoce konfigurovatelný, což nabízí velké možnosti použití. Jedno z možných použití je i poskytování TCP respektive UDP proxy. \cite{amoany_2020_getting}

SoCat je velmi známy a minimalistycký nástroj. Díky tomu je velmi jednoduchý na použití a vhodný na provoz v kontejneru. SoCat je zároveň dostupný, jako veřejný obraz kontejneru na platformě Docke~Hub. Z tohoto je velmi vhodným řešením pro provoz v rámci klastru kubernetes.

Další podmínkou pro vhodnou volbu proxy je možnost provozování proxy pro více spojení najednou. Pro tuto funkci nabízí SoCat přepínač fork. Při použití tohoto přepínače program vytvoří pro každé nové spojení nový proces, který bude dané spojení spravovat. Tímto je zajištěna podpora více spojení najednou.

Společně s jednoduchou konfigurací tak SoCat splňuje všechny požadavky a je vhodnou volbou pro provozování proxy v prostředí Kubenretes formou kontejneru. Následující část textu již bude předpokládat použití tohoto nástroje. 

\bigskip

SoCat přímo neumí pracovat s protokolem HTTP. V případě potřeby práce je možné využít některý z nástrojů, které tuto možnost nabízí. Vhodným kandidátem může být služba Nginx, kterou lze provozovat i jako proxy podporující HTTP. Nginx nabízí možnost proxy na čtvrté a sedmé vrstvě ISO/OSI modelu. Možnost řídit proxy pomocí HTTP protokolu pak nabízí velkou řadu možností pro rozšíření fungování proxy oproti řešením podporujíc pouze čtvrtou vrstvu ISO/OSI. Příkladem může být pokročilý load-balancing, podpora šifrování a kontrola routovaní pomocí HTTP. Pro účely práce nebude tento proxy server použit, z důvodu náročné konfigurace, která se nehodí v implementační části.      

\subsection{Nastavení sítě}
Důležitou částí návrhu řešení je i nastavení síťování v podu tak, aby podporoval více síťových rozhraní. Způsob, kterým lze tohoto dosáhnout je na teoretické úrovni popsán již v předchozí kapitole. Z tohoto textu bude návrh síťování vycházet. Pro podporu podu více síťových rozhraní v podu bude využit \textit{Kubernetes Network Custom Resource Definition De-facto Standard}.   

Tento standard deklaruje API rozhraní pro popis přídavných síťových rozhraní v rámci Kubernetes. Zároveň definuje základní podobu takzvaných gelegujících pluginů. Jako delegujícími pluginy jsou ve standardu označovány programy a řešení, které implementují zmíněný standard. Kubernetes Network Custom Resource Definition De-facto Standard odkazuje na část standardu CNI o delegování práce mezi moduly. Zmíněná část je krátce popsána v sekci \ref{cni}  první kapitoly. Právě díky této části je umožněna snadná implementace gelegujících pluginů. Jedním z těchto pluginů, je Multus CNI (Multus). 

%V kapitole o možných řešení je řešení pomoci proxy uvedeno, jako vyhovující a \textit{nejlepší možné}. Zároveň tato kapitola uvádí, že pro potřeby použití proxy je nutné umožnit tvorbu Podu s více rozhraními - tomuto přesně vyhovuje standart \textit{Kubernetes Network Custom Resource Definition De-facto Standard}.

%Contrail splňuje specifikaci CNI i Kubernetes Network Custom Resource Definition. Tento plugin splňuje potřeby pro řešení problému, jedná se ale o velký pokročilý plugin, který je těžké nastavovat a spravovat. Zároveň jeho použití může omezit některé specifické potřeby archytektur klaudu.

Multus je gelegujících plugin, který je vyvíjen přímo skupinou Network Plumbing Working Group, která spravuje zmíněný standard. Multus není přímo implementace CNI, projekt je označován jako takzvaný meta-plugin. \cite{hayashi_2019_multuscni}

Multus byl navržen jako zásuvný modul, který rozšiřuje základní Kubernetes CNI o funkci tvorby podu s více rozhraní. Této možnosti je dosaženo tím, že Multus umožňuje volání více implementací CNI standardu. Při použití to znamená, že Multus volá takzvaný \uv{hlavní} CNI modul (defaultní, pro umožnění komunikace v klastru) a následně \uv{vedlejší} modul, který do Podu vloží a nastaví přídavné rozhraní. Tímto způsobem je umožněno tvořit Pody s více síťovými rozhraními.. \cite{hayashi_2019_multuscni}

Způsob konfigurace přídavných síťových rozhraní je popsán již ve specifikaci Kubernetes Network Custom Resource Definition De-facto Standard. Více informací o standardu je popsáno v \ref{sec:kncrdds} části této práce.

Právě Multus je použit jako způsob umožnující tvorbu podu s Více rozhraními. Tento plugin byl, jelikož se jedná o velmi elegantní řešení. Poskytuje velkou volnost použití a nijak neomezuje síťování v Kubernetes. Multus je velmi dobrou volbou pro řešení problému.



%This is accomplished by Multus acting as a \uv{meta-plugin}, a CNI plugin that can call multiple other CNI plugins.\\Dokumentace uvádí termín \uv{hlavní} modul a \uv{delegovaný} modul. Hlavní modul je zodpovědný za síťování uvnitř interní sítě Kubernetes.\\deleguje konfiguraci a přidělení síťových rozhraní na další CNI zásuvné moduly. Tímto způsobem Multus CNI funguje jako takový orchestrátor pro ostatní CNI zásuvné moduly, umožňujíc současnou práci s více různými sítěmi. Toto chování je umožněno především tím, že specifikace CNI možnost delegace práce mezi pluginy. Toto je popsáno zmíněno v první kapitole v sekci popisující standard. Pro zjedudušení by se dalo říct, že při vzniku Podu multus vždy volá \uv{hlavního} CNI a pokud Pod definuje, tak i všechny delegované moduly. Tento způsob umožňuje tvorbu Podů s více síťovími rozhraními. Zároveň se jedná o velmi elegantní řešení, jelikož poskytuje velkou volnost použití a nijak neomezuje standartní síťování. Multus je velmi dobrou volbou pro řešení problému.  Konrétní použití je uvedeno v sekci \ref{prostredi}

\subsection{Ukázka nastavení proxy}\label{ukazkaProxy}
Pomocí služby SoCat a modulu Multus je již možné vytvořit funkční způsob pro komunikaci se zařízeními v privátních sítích. Pro snadnější pochopení, jakým způsobem je adresace v privátních sítích umožněna bude řešení ukázáno na konkrétním příkladu.

Následující příklad je nejednodušším případu užití. Cílem příkladu bude zprovoznit službu proxy přímo v systému kubernetes tak, aby poskytla komunikaci mezi interní sítí klastru a jedním zařízením v privátní sítě.

Jako první je potřeba vytvořit objekt Network~Attachment~Definition. Konkrérně se jedná o objekt s názvem \textit{bridge-conf}. Definice tohoto objektu je znázorněna ve výpisu kódu \ref{sample:nad}.

\begin{lstfloat}
\begin{lstlisting}[style=mybashstyle,
caption={Ukázka konfigurace Network Attachment Definition},
label={sample:nad}
]
apiVersion: k8s.cni.cncf.io/v1
kind: NetworkAttachmentDefinition
metadata:
  name: bridge-conf
spec:
  config: |-
    {
      "cniVersion": "0.3.1",
      "type": "bridge",
      "name": "bridge-conf",
      "bridge": "eb",
      "isGateway": true,
      "ipMasq": true,
      "ipam": {
        "type": "host-local",
        "ranges": [ [ { "subnet": "10.10.0.0/24"} ] ],
        "routes": [ {"dst": "0.0.0.0/0"} ],
        "dataDir": "/run/kube-edge/ipam-state"
      }
    }
\end{lstlisting}
\end{lstfloat}


Tento objekt jednoduše popisuje způsob, jakým bude vytvořeno přídavné síťové rozhraní v Podu obsahující proxy (běžící proces SoCat). Popis je obsažen v atributu objektu \verb|.spec.connection|. Tento atribut pak obsahuje kompletní popis a konfiguraci pro použití \uv{vedlejšího} pluginu CNI. Jako vhodným modulem pro ukázku byl zvolen CNI plugin bridge. Jedná se o stejný CNI plugin, který byl popsán v první kapitole této práce \ref{sitovaniKon}.

V tuto chvíli je vytvořen objekt popisující způsob vytvoření přídavného síťové rozhraní. Nyní je třeba vytvořit Pod, který bude provozovat potřebou službu proxy. Definice tohoto podu je znázorněna v ukázce \ref{sample:proxy}.

\begin{lstfloat}
\begin{lstlisting}[style=mybashstyle,
caption={Ukázka konfigurace podu s Proxy},
label={sample:proxy}
]
---
apiVersion: apps/v1
kind: Pod
metadata:
  name: sample-proxy
  annotations:
    k8s.v1.cni.cncf.io/networks: bridge-conf
spec:
  containers:
    - name: proxy-tcp
      image: alpine/socat
      args: [ "tcp4-listen:8080,fork", "tcp4-connect:172.17.16.120:8080" ]
\end{lstlisting}
\end{lstfloat}


Takto vytvořený pod bude provozovat proxy pomocí nástroje SoCat. Konfigurace SoCat je určena v objektu \verb|spec.containers|. Přídavné síťové rozhraní je deklarováno pomocí atributu \verb|.metadata.annotations.'k8s.v1.cni.cncf.io/networks'|, který odkazuj na již vytvořený Network Attachment Definition objekt. Pro kontrolu konfigurace je uveden výpis \ref{code:proxyValid}

\begin{lstfloat}
\begin{lstlisting}[style=mybashstyle,
caption={Ukázka nastavení proxy Podu},
label={code:proxyValid}
]
# Exec /bin/sh inside Pod
[1]$ kubectl exec -it sample-proxy -- sh

# Show SoCat process (pid=1)
[2]$ ps -p 1 -o args
>>> COMMAND
>>> socat tcp4-listen:8080,fork tcp4-connect:172.17.16.120:8080

# List network interfaces and IP addresses
[2]$ ip -br a
>>> lo        UNKNOWN 127.0.0.1/8 ::1/128 
>>> eth0@if38 UP      10.244.2.17/24
>>> net1@if39 UP      10.10.0.32/24
\end{lstlisting}
\end{lstfloat}


V uvedené ukázce lze vidět, že v Podu běží proces SoCat, který provozuje službu proxy. Zároveň je možné vidět, že Pod obsahuje celkem tři síťová rozhraní. Prvním rozhraní je typu \textit{loopback} toto rozhraní je obsaženo v každém podu pro potřeby komunikace v rámci samotného podu. Druhým rozhraním je \verb|eth0@if38| toto rozhraní je vytvořeno CNI modulem flannel a složí pro komunikaci uvnitř klastru. Posledním rozhraním je \verb|net1@if39|. Toto zařízení je vytvořeno díky meta-pluginu Multus. Konkrétně se jedná o  rozhraní vytvořene \uv{vedlejším} modulem bridge, jak je definováno v objektu Network Attachment Definition.

Uvedená konfigurace korektně nastavuje proxy tak, aby spojení, které přijdou na otevřený port 8080 daného podu, byly přeposlány dále. V uvedeném případě, na zařízení (s IP adresou \verb|172.17.16.120|) v privátní síti, dostupné z pracovního uzlu - kde pod běží. O správně doručení paketů komunikace se postará již hostující systém Podu. Samotné opuštění interní sítě klastru je umožněno díky přidanému síťovému rozhraní \verb|net1@if39|.

Uvedený příklad vyžívá CNI pluginu bridge jako\uv{vedlejším} plugin pro nastavení síťování. Takto navržené použití podporuje i libovolné jiné implementace CNI standardu. Díky tomuto, řešení poskytuje úplnou flexibilitu při konfiguraci sítě, jelikož lze použít libovolný CNI plugin.

\bigskip
Tento příklad demonstruje použití nástroje SoCat a meta-pluginu Multus tak, aby umožnila komunikaci objektů v klastru se zařízeními v přilehlé privátní síti. Komunikaci určenou pro zařízení adresovat na takto vytvořený Pod. Pod bude plně abstrahovat komunikaci se zařízením. V reálné případu užití se hodí využít navíc objektu \textit{Service} odkazujíc na daný pod. Tímto lze docílit zpřístupnění standardních funkcí jako je například load-balancing a podpora DNS.

Následující a poslední sekce této práce se bude integrací uvedeného řešení do prostředí kubernetes a automatizací konfigurace pro snadnější použití.  

\subsubsection*{Podpora obousměrnosti komunikace}
Zmíněná ukázka poskytuje jednosměrné řešení umožnující komunikaci se zařízeními v privátní síti. Adresace směrem do klastru z privátní sítě není podporována. Toto odpovídá omezení zmíněné v sekci o vymezení implementace.

V případě potřeby adresace do vnitřní sítě klastru se nabízí dvě možnosti. První z nich je použití objektu Ingress. Pomocí Ingress lze dosáhnout možnosti adresace do sítě klastru. Takové řešení je validní, ale přináší nemalé nároky na konfiguraci. Pro podporu více privátních síťových segmentů je použití Ingress neefektivní a pro automatizaci nevhodné.

Druhým způsobem je možnost využití upraveného navrženého řešení. Pro realizaci je možné využít proxy stejným způsobem, jako popsáno výše, jen v opačném směru. Proto aby byla komunikace umožněna musí být vhodně zvolen \uv{vedlejším} CNI plugin. Příklady pluginů, které lze využít jsou host-device a macvlan z množiny referenčních modulu, které jsou vyvíjeny v rámci specifikace CNI. Pro správné použití může být třeba povolit přeposílání ARP paketů na edge uzlech.

V případě implementace podpory obousměrné adresace se nabízí i poskytnout DNS překlad doménových jmen na edge uzlech, pro usnadění adresování. Provozovat takové službu pro objekty interní sítě kubernetes je snadno realizovatelné pomocí CoreDNS s případně upraveným Kubernetes pluginem.  

\section{Rozšíření Kubernetes}\label{sec:extend}
Kubernets je velmi dobře navržen proto, aby byl jednoduše rozšiřitelný. Na oficiálních stránkách projektu Kubernets je uvedeno: 
\begin{displayquote}
\uv{Kubernetes is highly configurable and extensible. As a result, there is rarely a need to fork or submit patches to the Kubernetes project code.}\\(Kubernetes je vysoce konfigurovatelný a rozšiřitelný. Díky tomu je jen zřídka potřeba kód projektu Kubernetes forkovat nebo zasílat záplaty.) \cite{thekubernetesauthors_2022_kubernetes}
\end{displayquote}

V případě potřeby, Kubernetes nabízí způsoby, kterým lze systém rozšířit. Dokumentace projektu popisuje různé potřeby pro rozšíření systému a zároveň odkazuje na způsoby, jak toho dosáhnout. Pro účely této práce je důležitá sekce popisující rozšiřování Kubernetes API, automatizace práce a rozšíření síťování.\textbf{CITOVAT}

%\subsection{Rozšíření síťování - CNI}
%Pro potřeby rozšíření síťování Kubernetes doporučuje tvorbu vlastního CNI modulu dle CNI specifikace. Tento způsob dává dobrý smysl. Tvorbou vlastního CNI modulu lze plně kontrolovat síťování v Kuberneets. To přináší velkou svobodu a možnost přizpůsobení konkrétním potřebám.

%Při vývoji vlastního CNI pluginů je nutné plnit všechny požadavky stanovené specifikací a zároveň plnit požadavky, které jsou popsány na stránkách Kubernetes. Většina implementací se skládá z CNI konfiguračního souboru, binárního spustitelného souboru a běžící aplikace v systému Kubernets. Pro správné fungování CNI pluginu je nutné zajistit následující. Daná specifikace se musí nachází na každém pracovním uzlu v adresáři \verb|pod.spec.hostNetwork|. Adresář s CNI pluginy (typicky \verb|/opt/cni/bin|) musí obsahovat potřebný spustitelný soubor specifikovaný v konfiguračním souboru. Na každém pracovním uzlu musí běžet aplikace, která pomáhá se síťování v klastru. Těchto podmínek lze jednoduše docílit pomocí standartních objektů Kubernets. \cite{bigelow_2022_explore}\\

%\subsubsection{Zvolení potřebného delegujícího CNI pluginu}\label{sub:multus}


\subsection{Rozšiřování Kubernetes API - CRD}\label{CRD}
Velká část systému Kubernetes se skládá z definic objektů. Kubernetes objekty jsou prvky, které uchovávají stav klastru, lze si je představit jako datové struktury nesoucí informace. Tyto objekty slouží i pro komunikaci s Kubernetes API serverem. API server poskytuje základní CRUD operace k těmto objektům, díky těmto operacím lze konfigurovat, nastavovat a ovládat samotný klaud.

Kubernetes poskytuje základní objekty pro práci s klastrem. Příklady těchto objektů jsou Pod, Deployment, Endpoint, Namespace... Tyto objekty jsou navrženy a spravovány autory Kubernetes. Jedním ze standartích objektů je i CustomResourceDefinition. CustomResourceDefinition (zkráceně CRD) je meta-objekt, který umožňuje definovat vlastní nové objekty. CRD pak umožňuje definovat strukturu nových objektů (datových struktur). Příkladem použití CRD je již zmíněný Network Attachment Definition, který byl definovaný v dokumentu Kubernetes Network Custom Resource Definition De-facto Standard. V případě, že je nový CRD objekt vytvořen, je tento objekt přidán do Kubernetes API a tím je možné s ním Kubernetes pracovat. Pro nový datový objekt jsou automaticky vytvořeny základní CRUD operace.

CustomResourceDefinition poskytuje velmi elegantní způsob a jakým lze rozšiřovat Kubernetes API o nové objekty.  
\subsubsection{Návrh vlastních CRD objektů}\label{CRDDEF}
Jedním z požadavků zmíněných v sekci \ref{sec:pozadavky} je Jednoduchost. Pro jednoduchost je požadováno, aby řešení podporovalo zavedené způsoby komunikace s klastrem. Toho lze velmi dobře dosáhnout právě za použití CRD. Navržením nových objektů rozšíří standartní API Kubernetes dobře rozšířit API Kubernetes. S nově vzniklými objekty lze následně pracovat pomocí operátorů (popáno v další sekci). Tento způsob zároveň přináší řadu výhod a usnadnění. Jelikož se o správu objektů stará přímo Kubernetes, můžeme se spolehnout na persistenci dat, která vychází z vlastností databáze etcd, tato databáze bude řešit i kolize, které mohou vzniknout při souběžném přístupu k datům. Další výhoud je, že tyto objekty budou dokumentovány standartním způsobem kuebrneets. O jejich validaci se z velké části bude starat Kubernetes API server. Jelikož tento způsob nabízí celou řadu výhod a zároveň splňuje požadavek pro jednoduchost, je vhodným pro účely řešení problému.

Nyní se zaměříme pouze na návrh CRD objektů pro účely řešení problému. Tyto objekty budou tvořit kompletní API pro ovládání tvořeného rozšíření Kubernetes.

Navrhované API bude obsahovat tři CRD objkty. Prvním objektem bude \textit{Network Attachment Definition}. \textit{Network Attachment Definition} je deinován v standardu \textit{Kubernetes Network Custom Resource Definition De-
facto Standard} popsaném v sekci \ref{sec:kncrdds}. Pro více informací o tomto objektu doporučuji si přečíst samotnous specifikaci. Dálšími dvěma objety budou Device a Connection. Oba objekty jsou součástí API skupiny \textit{edge-operator.k8s.dvojak.cz}.  
\subsubsection*{Device}
Prvním z objektu rozšiřující standartní API je \textit{device}. Objekt device reprezentuje zařízení, která se nachází v privátní síti, mimo klastr. Toto zařízení by mělo být dostupné přes jeden či více z uzlů v clusteru, tyto uzly pak souží jako brány k zařízení. Ve schémau \ref{fig:schema} z přechozí kapitoly by zařízení byly použity pro uchování informací o \textit{device01-01} a \textit{device01-02}. Objekt složí primárně pro ukládání informací o zařízeních. Ve výpisu kódu \ref{sample:device} lze vidět ukázkovou definici objktu.
\begin{lstfloat}
\begin{lstlisting}[language=bash, caption={Ukázka CRD device}, label={sample:device}]
---
apiVersion: edge-operator.k8s.dvojak.cz/v1
kind: Device
metadata:
  name: device01-01
spec:
  nodeName: kedge1
  up: true
  ipAddress: 172.17.16.120
  components:
    - name: backend
      up: true
      handlers:
        - name: api server
          protocol: HTTP
          port: 443
          endPoints:
            - https://<path>/api/*
            - https://<path>/version
...
\end{lstlisting}
\end{lstfloat}
\begin{description}
    \item \verb|.spec.nodeName| --- Název pracovního uzlu, který je připojen do privátní sítě a sítě Kuberetes. Tento uzel bude použit pro vytvoření mostu a provozování proxy.
    \item \verb|.spec.up| --- Definuje, zda je dané zařízení zapnuté a může být použito.
    \item \verb|.spec.ipAddress| --- IP adresa zařízení v privátní síti.
    \item \verb|.spec.components| --- Seznam komponent, které na zařízení běží a jsou dostupné po síti. Komponenta označuje běžící aplikaci. Příkladem takové komponenty může být webový server či jiná aplikace.  
    \item \verb|.spec.components.name| --- Název komponenty pro identifikaci.
    \item \verb|.spec.components.up| --- Určuje, zda je daná komponenta zapnuté a může být použita.
    \item \verb|.spec.components.handlers| --- Seznam portů, které daná aplikace obsahuje.
    \item \verb|.spec.components.handlers.name| --- Název portu. Tento parametr nemá přímé použití, slouží pouze pro dokumentační účely. 
    \item \verb|.spec.components.handlers.protocol| --- Protokol značuje typ použitého protokolu, povolené hodnoty jsou TCP, UDP a HTTP
    \item \verb|.spec.components.handlers.port| --- Číslo daného portu
    \item \verb|.spec.components.handlers.endpoints| --- Seznam koncových bodů (endpointů). Tento seznam je nepovinný a slouží pouze pro dokumentační účely.
\end{description}

Od této chvíle budeme zařízení v privání síti značit jako \textit{Device}.

\subsubsection*{Connection}
Druhým deinovaným CRD objektem je \textit{Connection}. \textit{Connection} reprezentuje již vytvořené spojení se zařízením v privátní síti \textit{Device}. Objekt složí primárně pro uchovávání informací o spojení a zároveň k ovládání daného rozšíření. Infomace o ovládání bude popsána v sekci \ref{sec:operator} o operátoru. Struktura objektu je nastíněna v ukázce kódu \ref{sample:connection}.


Connection represents a connection between a device and Kubernetes cluster. By creating a connection, Service will be created for abstracting device connection. By sending data to that Service, data will be sent to device.

\begin{lstfloat}
\begin{lstlisting}[style=mybashstyle, caption={Ukázka CRD connection}, label={sample:connection}]
---
apiVersion: edge-operator.k8s.dvojak.cz/v1
kind: Connection
metadata:
  name: sample-connection
  namespace: default
spec:
  deviceName: device01-01
  networkName: bridge-conf
  componentNames:
    - backend
\end{lstlisting}
\end{lstfloat}
\begin{description}
    \item \verb|.spec.deviceName| --- Název objektu \textit{device}, ke ketrému bude vytvořeno spojení proxy. 
    \item \verb|.spec.networkName| --- Návez \textit{NetworkAttachmentDefinition}. Tento objekt definuje nasatveeí sítě v Podu, které bude použito pro Pod s proxy.
    \item \verb|.spec.componentNames| --- Seznam komponent objektu \textit{device}, ke ketrému bude vytvořeno spojení proxy. 
\end{description}

\subsection{Automatizace práce - Operátor}\label{sec:operator}
Objekty slouží jako datové struktury pro ukládaní informací o stavu klatru. Tyto objekty jsou zpracovávány pomocí Kubernetes kontrolerů. Kontrolery jsou nekonečné smyčky, které mají za úkol udržovat klastr v požadovaném stavu, který je definovaný pomoc9 objektů uložených v \textit{etcd} databázi. Kontrolery lze chápat jako pracovníky, kteří spravují celý systém Kubernetes. Oficiální dokumentace uvádí příklad na termostatu. Termostat je typicky nastaven na fixní požadovanou teplotu, kterou má za úkol v místnosti udržet. V případě, že teplota klesne Pod požadovanou hladinu, pak provede potřebné kroky pro zvýšení teploty v místnosti na požadovanou teplotu. Stejným způsobem fungují kontrolery v systému Kuberneets. Neustále porovnávají aktuální stav klastru se stavem požadovaným (definovaným objekty). V případě že se tyto dva stavy liší, pak se pokusí aktuální stav klastru co nejvíce přiblížit stavu požadovanému. Tyto kontrolery jsou najčastěji provozovány formou malých aplikací.

Funkce kontroleru je pěkně ilustrována na následujícím příkladu zdrojového kódu. \cite{nguyen_2017_a}
\begin{lstfloat}
\begin{lstlisting}[language=R, caption={Ukázka CRD device}, label={sample:controller}]
for {
  desired := getDesiredState()
  current := getCurrentState()
  makeChanges(desired, current)
}
\end{lstlisting}
\end{lstfloat}

Stejně jako Kubernetes nabízí základní skupinu standardních objektů, tak nabízí i standardní množinu kontrolerů. Tyto kontrolery jsou spravovány vývojáři Kubernetes a jsou součástí základní instalace systému. Příkladem těchto kontrolerů jsou Deployment Controller, Endpoint Controller, Namespace controler\ldots Zmíněný výčet je velmi podobný výčtu objektů v předchozí kapitole. Je tomu tak, jelikož zmíněné kontrolery implementují logiku pro práci právě s danými objekty. Kontrolery společně s objekty tvoří základní stavební bloky Kubernetes, které představují způsob, jakým je Kubernetes navržen. Tento koncept lze přirovnat k návrhovému vzoru, který usnadňuje správu a škálování aplikací v kontejnerech prostřednictvím abstrakce a deklarativního přístupu. Tento způsob zároveň umožňuje snadnou rozšiřitelnost.

V předchozí částí byl zmíněn způsob rozšiřování Kuberneets API pomocí CRD. Právě s CRD úzce souvisí návrhovým vzor operátor. Operátor je návrhový vzor definovaný Kubernetes, který umožňuje vytvářet moduly podobné kontrolerům. 
\subsubsection{Návrhový zvor Operátor}
Operátor je aplikace, která rozšiřuje funkcionalitu Kubernetes o specifické vlastnosti. Velmi často pracuje s Custom Resource Definitions (CRD). Operátor navazuje na principy kontrolerů v Kubernetes a zavádí možnst automatizace pro správu aplikací v prostředí Kubernetes. Hlavní myšlenkou operátoru je umožnit automatizace práce administrátorům aplikací. Operátor lze označit jako modul, který přidává velmi specifické funcionality do systému Kubernetes. 

Nejčastěji je operátor tvořen kombinací CRD a kontroleru, který zpracovává a spravuje instance těchto objektů. Vetšina operátorů implementuje respektivě navrhuje jedno či více rozhraními, které Kuberneets nabízí. Těmito rozhraními jsou:  
\begin{itemize}
    \item Custom Resource Definitions (CRD)\\
    Objekty umožňující uchování dat o klastru a komunikaci s operátor pomocí Kubernetes API
    \item Kontroller\\
    Při implementaci kontroleru (nekonečné smyčky) je implementace často abstrahována do implmentací třech základních funkcí.
    \begin{itemize}
        \item Reconcile\\
        Reconcile je funkce, která udržuje stav klatv klastru dle pořadované funkcionality, kterou operátor implementuje. Tato funckc je volána, při vzniku pozorovaného CRD, pozorovaný CRD byl změnen, nebo byl vytvořen požadavek na spustění této funkce. 
        \item Status Modified\\
        Status Modified je funkce implementující, logiku, která je přováděna ve chvíli změny vnitřního stavu objektu. Funkce je volána při změně vnitřího stavu pozorovaného CRD objektu.
        \item Deleted\\
        Posledním funkcí abstrahující kontroller je funkce deleted. Funkce je použita pokaždé, kdž je pozorovaný CRD objekt změněn.
    \end{itemize}
    \item Finalizers\\
    Finalizers velmi úzce souvisí s implementací kontrolerů. Jedná se o logiku, která umožňuje dokončit a začistit prováděné operace. Finalizer je typicky volán, při neočekávané chybě systému a ihned po smazání objektu. Jedná se o způsob, kterým lze udržovat nastavení kalstru v co nejvíce konzistentím stavu.
    \item Webhooks\\
    Webhooks je zp;sob, kter7m je možné se napjit na slogiku stávajícího API server a uvlivnit tak spracování dat při používíní API serveru. Kubernetes umožňuje implementovat dva zaákladní typy.
    \begin{itemize}
        \item Mutator\\
        Mutator je prvním způsobem, jak ovlivnit fungování API serveru. Mutator umožňuje modifikovat požadavek na API server. Toto umožňuje aplikovat různé polityky na systém, případně opravid nekonzistence dat. V prostředí kubenretes se jedná o velmi mocný nástroj, jak ovlivnit chovhání uživatelů.  
        \item Validator\\
        Validator je druý způsob, jak lze v prostředí kubenetes ovlivnit příjmání pořžadavku. Tento webhook umožňuje validovat příchozí požadadavky. V případě, že data neodpovídají předem stanoveným pravidlům nebo očekávaným hodnotám, je možné tímto webhookem odmítnout daný požadavek.
    \end{itemize}
\end{itemize}
Návrhový vzor operátor je velmi pěkně navrhnut a rozmyšlen, pro detailnějsí pochopení vzoru doporučuji blog \href{https://iximiuz.com/en/posts/kubernetes-operator-pattern/}{\textit{Exploring Kubernetes Operator Pattern}}, oficiální dokunetaci Kubernetes a předášku \href{https://youtu.be/KBTXBUVNF2I}{\textit{Tutorial: Zero to Operator in 90 Minutes!}} od \textit{Solly Ross}
 \cite{bhler_2021_kubeops},
 \cite{kaplan_2022_argocd},
 \cite{velichko_2021_exploring},
 \cite{cncfcloudnativecomputingfoundation_2020_tutorial}

\subsubsection{Oprátor edge-operatrot}
Poslední částí implementace je realizace operátoru, který bude pracovat s výše vydefinovanými CRD a automatizovat tvorbu proxy Podu, jak je ukázáno v sekci \ref{ukazkaProxy}. Od této chvíle bude tento operátor označován jako edge-operator.

Oficiální dokumentace Kubernetes vybízí k použiti některé z frameworků pro implementaci operátorů. frameworků pro realizaci operátoru je celá řada, Kubernetes na svých stránkách uvádí tyto příklady: \textit{kubebuilder} (GO), \textit{KubeOps} (.NET), \textit{kube-rs} (Rust) a \textit{Java Operator SDK}. Pro implementaci operátoru edge-operatrot byl použit \uv{KubeOps - The dotnet Kubernetes Operator SDK}.\footnote{https://github.com/buehler/dotnet-operator-sdk}

KubeOps je framework obsahující sadu nástrojů pro implementaci operátoru v prostředí .NET. Jedná se o udržovaný projekt, který je stále vyvíjen. Jeho autor udává, že projekt je silně inspirován projektem kubebuilder. KubeOps je implementován v nejnovější verzi freamworku s dlouhodobou podporou \verb|.NET 7|. Implementace KubeOps hojně využívá technologii \verb|APS.NET|, která slouží pro vývoj webových služeb. Použití zmíněné technologie pro tvorbu webu jako základ pro realizaci operátorů dává velký smysl, jelikož právě \verb|APS.NET| je dle průzkumu \textit{Stack Overflow Developer Survey 2022} nejpoužívanější technologií pro vývoj webových backendových aplikací profesionálním vývojáři. \footnote{https://survey.stackoverflow.co/2022/\#most-popular-technologies-webframe-prof} Toto velmi zpřístupňuje možnost implementace operátorů.

V následujících sekcích bude krátce vysvětlen mechanizmus fungování operátoru edge-operator a stručně nastíněn způsob, kterým byl operátor implementován pomocí KubeOps freamworku.

\subsubsection*{CRD}
První s čím se uživatel operátoru setká jsou CRD Device a Connection. Jedná se o objekty, které definovány v sekci \ref{CRDDEF}. Tyto objekty jsou v operátoru definovaný pomocí tříd, které korespondují navrženému API. Mimo standardní datové typy a kontrukce, které nabízí jazyk \verb|C#|, jsou k definici použity i atributy (anotace), které nabízí freamwork KubeOps. Tyto atributy slouží převážně k dodefinování metadat o objektu, jako je dokumentace a podoba specifikace výseledných CRD.

Objekty jsou definovány pomocí tříd \verb|DeviceEntity| a \verb|ConnectionEntity|, které dědí příslušné třídy z KubeOps.

\subsubsection*{Validátor}
Edge-operator implementuje validátor pro každý ze zmíněných objektů. Oba tyto validátory slouží pro kontrolu správnosti definic objektů. Tato kontrola probíhá vždy, když je jeden z objektů vytvořen, nebo modifikován. Kontrolované vlastnosti objektů jsou detailně popsány v dokumentaci projektu. 

Ve výchozím nastavení, operátor zamítne požadavek pro vytvoření respektive editaci objektu v případě, že objekt nesplní jednu z kontrolovaných vlastností. V případě potřeby je možné operátor přepnout do ne-striktního módu validace. V takovém případě, kdy je porušena některá z požadovaných vlastností objektu, operátor pouze informuje o jaká porušení se jedná a požadavek nezamítne. Po provedení požadavku jsou data propsána do interní Kubernetes databáze etcd. Tento mód není určen pro reálné nasazení, jelikož může docházet k nedefinovanému chování. To vede k nekonzistenci stavu klaudu, tyto nekonzistence můžou vyžadovat přímí zásah administrátora do vnitří logyky fungování operátoru.

Validátory jsou realizovány pomocí tříd \verb|DeviceEntityValidator| a\\\verb|ConnectionEntityValidator|. Tyto třídy implementují rozhraní dodávaná KubeOps. Realizace validátorů je umožněna díky takvaných webhooks, která Kubernetes poskytuje. V zásadě se jedná o callback, který je volán komponentou kube-api. 

\subsubsection*{Kontroler}
Hlavní částí edge-operator kontroler objektu \textit{Connection}. Implementace tohoto kontroleru automatizuje vytváření služeb proxy, která umožňují samotnou komunikaci. Jedná se o automatizaci kroků, které jsou uvedeny v sekci \ref{ukazkaProxy} této kapitoly.

Samotný kontroller implementuje metoody Reconcile a Delete kontrolleru. Tyto metody se spoléhají na korektní definicy CRD objektů, pro správné fungování je vyžadováno, aby databáze etcd obsahovala objekty, které jsou validní dle výše popsaného validátoru.

Reconcile metoda implementuje logiku pro vytváření nový spojení a udržování již existujících spojení (Connection). Při vzniku nového spojení (vytvoření objektu Connection) metoda vytvoří obejkt Deployment zaobalující objekt Pod, který slouží jako proxy služba do privátní sítě k danému zařízení. Pro každé spojení je vytvořen právě jeden Pod. Každá handler definovaný v objektu Device odpovídá jednomu kontejneru běžícímu ve vytvořeném podu. Mimo deployment vyváří kontroller i objekt Service, díky které umoňuje adresovat daná zařízení pomocí DNS jména. Registrované DNS jméno zařízení, ve výchozím konfiguraci, odpovídá \verb|dop-<name_of_service>|. Vystavené porty objektem Service odpovídají portům na daném zařízení. 

Druhou implementovanou funkcionalitou je Delete. Tato metoda je volána vždy, když dojde ke smazání objektu Service.

Kontroller je konfigurovatelný, díky tomuto je možné upravit jeho chování konkrétním potřebám užití. Implementace kontrolleru je součástí třídy \verb|ConnectionController|. 

\subsubsection*{Instalace a ukázka použití operátoru}
Zdrojový kód operátoru edge-operator je veřejně dostupný na platformě GitHub v repozitáři \verb|dvojak-cz/Bachelor-Thesis|. Teto repozitár obsahuje mimo zkdojvého kódu i definice prostředí (popsáno výše), dokumentaci k operátoru a vše ostatní spjaté s touto prací.

Pro jednoduché nasazení a použití operátoru lze využít dokumentaci na webu\\\href{https://bt.project.dvojak.cz/}{\textit{bt.project.dvojak.cz}}. Instalace operátoru je umožněna pomocí sady konfigurovatelných manifestů, které jsou součástí každého release daného repoziáře. Potřebné kontejnery pro testování a instacaci operátoru jsou dosupné v kontejnerovém repozitáři GitHub.

Zmíněná dokumentace obsahuje i příklad použití samotného kontrolleru. Definice objektů jsou dosupné v repozitáři.  
