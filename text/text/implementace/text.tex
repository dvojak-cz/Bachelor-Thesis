\chapter{Implementace}
\noindent\makebox[\linewidth]{\rule{\paperwidth}{0.4pt}}
\begin{chapterabstract}
Tato kapitola popisuje nastíněného řešení problému z předchozí kapitoly a implementaci operátoru, rozšiřující API kubernetes tak, aby automatizovalo konfiguraci navrženého řešení. Cílem této kapitoly je vytvořit modul pro systému kubernetes, který umožní jednoduchý způsob pro propojení interní sítě kubernets a privátní sítě přilehlé k klastru.
\end{chapterabstract}

\section{Rozšiřování k8s}
Kubernets je velmi dobře navržen proto, aby byl jednoduše rozšiřitelný. Na oficiálních stránkách projektu kubernets je uvedeno: 
\begin{displayquote}

Kubernetes is highly configurable and extensible. As a result, there is rarely a need to fork or submit patches to the Kubernetes project code.
\end{displayquote}
\noindent\rule{2cm}{0.4pt}
\begin{displayquote}
Kubernetes je vysoce konfigurovatelný a rozšiřitelný. Díky tomu je jen zřídka potřeba kód projektu Kubernetes forkovat nebo zasílat záplaty.
\end{displayquote}
\footnote{https://kubernetes.io/docs/concepts/extend-kubernetes/}\footnote{deepl}

V případě potřeby kubernetes nabízí způsoby, kterým lze systém rozšířit. Dokumentace projektu popisuje různé potřeby pro rozšíření systému a zároveň odkazuje na způsoby, jak tyto potřeby řešit. Pro účely této práce je důležitá sekce popisující rozšiřování kubernetes API, automatizace práce a rozšíření síťování.

\subsection{Rozšíření síťování - CNI}
Pro potřeby rozšíření síťování kubernetes doporučuje tvorbu vlastního CNI modulu dle CNI specifikace. Tento způsob dává dobrý smysl, tvorbou vlastního CNI modulu lze plně kontrolovat síťování v kuberneets. To přináší velkou svobodu a možnost přizpůsobení konkrétním potřebám.

Při vývoji vlastního CNI plugunů je nutné plnit všechny požadavky stanovené specifikací a zároveň plnit požadavky, které jsou popsány na stránkách kubernetes. Většina implementací se skládá z CNI konfiguračního souboru, binárního spustitelného souboru a běžící aplikace v systému kubernets. Pro správné fungování CNI pluginu je nutné zajistit následující. Daná specifikace se musí nachází na každém pracovním uzlu v adresáři \verb|pod.spec.hostNetwork|. Adresář s CNI pluginy (typicky \verb|/opt/cni/bin|) musí obsahovat potřebný spustitelný soubor specifikovaný v konfiguračním souboru. Na každém pracovním uzlu musí běžet aplikace, která pomáhá se síťování v klastru. Těchto podmínek lze jednoduše docílit pomocí standartních objektů kubernets. \footnote{https://www.techtarget.com/searchitoperations/tip/Explore-network-plugins-for-Kubernetes-CNI-explained}\\

\subsection{Rozšiřování kubernetes API - CRD}
Velká část systému kubernetes se skládá z definic objektů. Kubernetes objekty jsou prvky, které uchovávají stav klastru, lze si je představit jako datové struktury nesoucí informace. Tyto objekty slouží i pro komunikaci s kubernetes API serverem. API server poskytuje základní CRUD operace k těmto objektům, díky těmto operacím lze konfigurovat, nastavovat a ovládat samotný klaud.

Kubernetes poskytuje základní objekty pro práci s klastrem. Příklady těchto objektů jsou Pod, Deployment, Endpoint, Namespace... Tyto objekty jsou navrženy a spravovány autory kubernetes. Jedním ze standartích objektů je i CustomResourceDefinition. CustomResourceDefinition (zkráceně CRD) je meta-objekt, který umožňuje definovat vlastní nové objekty. CRD pak umožňuje definovat strukturu nových objektů (datových struktur). Příkladem použití CRD je již zmíněný Network Attachment Definition, který byl definovaný v dokumentu Kubernetes Network Custom Resource Definition De-facto Standard. V případě, že je nový CRD objekt vytvořen, je tento objekt přidán do kubernetes API a tím je možné s ním kubernetes pracovat. Pro nový datový objekt jsou automaticky vytvořeny základní CRUD operace.

CustomResourceDefinition poskytuje velmi elegantní způsob a jakým lze rozšiřovat kubernetes API o nové objekty.  
\footnote{ten clanek kter citoval hezkej pan}
\subsection{Automatizace práce - Operátor}
Objekty slouží jako datové struktury pro ukládaní dat o klatru. Tyto objekty jsou spracovávany pomocí kubernetes kontrolerů. Kontrolery jsou nekonečné smyčky, které kontrolují mají za cíl udrřovat klastru v požadovaném stavu, který je definovaný objekty. Kontrolery lze chápat jako pracovníky, kteří spravují celý systém kubernetes. Oficiální dokumentace uvádí příklad na termostatu. Termostat je typicky nastaven na fixní požadovanou teplotu, kterou má za úkol v místnosti udržet. V případě, že teplota klesne pod požadovanou hladinu, pak provede potřebné kroky pro zýšení teploty v místnosti na požadovanou teplotu. Stejným způsobem fungují kontrolery v systému kuberneets. Neustále porovnávají aktuální stav klatrtu se stavem požadovaným (definovaným objekty). V případě že se tyto dva stavy neshodují, pak se pokusí aktualní stav co nejvíce přiblížit stavu požadovanému. Tyto kontrolery jsou najčastěji provozovány fromou maých aplikací.

Funkce kontroleru je pěkně ilustrována na následujícím příkladu zdrojového kódu.\footnote{https://docs.bitnami.com/tutorials/a-deep-dive-into-kubernetes-controllers/\#controller-pattern}
\begin{verbatim}
for {
  desired := getDesiredState()
  current := getCurrentState()
  makeChanges(desired, current)
}
\end{verbatim}

Stejně jako kubernetes nabízí základní množinu standardních objektů, tak nabízí i standardní množinu kontrolerů pro tyto objekty. Tyto kubernetes jsou spravovány vývojáři kubernetes a jsou součástí základní instalace systému. Příkladem těchto kontrolerů jsou Deployment Controller, Endpoint Controller, Namespace controler \ldots Zmíněný výčet je velmi podobný výčtu objektů v předchozí kapitole. Je tomu tak, jelikož zmíněné kontrolery implementují logiku pro práci s danými objekty.   

V předešlém případě byl zmíněn způso rozšiřování kuberneets API pomocí CRD. Právě s CRD úzce souvisí kuberneets návrhovým zvorem operátor. Operátor je návrhový vzor definovaný kubernetes, který umožňuje vytvářet moduly podobné kontrolerům. Primární účel návrhového vzoru je poskytnout způsob, jakým jednoduše vyvíjet rozšíření pro systém kubernetes. Tato rozšíření často poskytují automatizaci procesů a správu aplikací běžících k kubernetes. Operátor (modul splňující návrhový vzor) je velmi často používán pro spravování CRD objektů, při takovém použití lze  operátor nazývat i kontrolerem. Příkladem operátoru může být ArgoCD operátor, který automatizuje konfiguraci klastru z gitových repozitářů.
\footnote{soc, kaplan -- insatll-fest -- argoCD, hezkej pan}\footnote{https://iximiuz.com/en/posts/kubernetes-operator-pattern/}

\textbf{Jak funguje operator?}

Návrhový vzor operátor poskytuje dobrý způsob jakým rozšiřovat samotné fungování kubernetes.
\section{Implementace}
\section{Ukázka}
